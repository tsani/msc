\chapter*{Preface}
\addcontentsline{toc}{chapter}{Preface}

\section*{Ackowledgements}

Although a thesis is something one writes primarily alone, it documents work
that would be impossible without the large support structure that comes with a
research group in a university. In my case, I am truly grateful to Complogic
group at McGill, and I am thankful that so many of its members have helped me
through the years of work poured into my research.

I want to especially acknowledge David Thibodeau, desk neighbour and colleague
extraordinaire. His insight in resolving both practical implementation concerns
with Harpoon as well as more theoretical issues was invaluable.
I very much appreciated his persistent readiness to hear out my crazy ideas and
to offer his own crazy ideas in return.
I look back fondly on the summer reading group we put together for Homotopy Type
Theory and the many times we got sidetracked messing around with Agda.
I will consider myself very lucky should I ever have an officemate so equally
both fun and intellectually stimulating as him.

Next I want to thank Clare Jang and Marcel Goh for their help in more directly
developing Harpoon. Clare took charge of several tricky implementation tasks
such as Harpoon's ability to overwrite existing holes in proof scripts with new
proof fragments. Marcel helped to develop the theory for translation from proof
scripts to Beluga programs as part of an undergraduate research course, and he
wrote the preliminary code for typechecking and translation proof scripts.

Of course, I want to recognize next the tremendous support of my advisor,
Brigitte Pientka. I vividly recall still to this day a time when I told a friend
in first year that I would be so happy to do a summer research project with
Brigitte. Little did I know then that I would find her so engaging and
stimulating that I would do an entire master's degree under her supervision.
Beluga is a quite the large project, so it is entirely due to her guidance that
I was able to understand it so deeply and to make improvements to it.
Her careful attention to detail and her principled approach to programming
language theory taught me a lot about how one should conduct research.
I admire her involvement in the research community, and I aspire to myself be so
involved in the communities I should find myself in.

Finally, I want to thank my partner Eric. I especially want to thank him for
being so understanding of the time commitments that graduate school involves. I
reminisce often about our walks, talking about logic, proofs, and theory. Few
people outside this specific research community have much interest in what I do,
so it was crucial for me to have an outsider to bounce ideas of off and to keep
me sane and grounded.

\section*{Contribution of authors}

The main technical exposition in Chap.~\ref{chap:theory} is primarily my own
original work. Prof.~Pientka gave me a lot of guidance early on for ideas on how
to define the structured proof script language and she helped me to develop its
type system. She also guided me in the implementation as I brushed up against
several complex features of Beluga.

I developed the formalization of the interactive actions themselves, in
Sec.~\ref{sec:harpoon-actions} which led me to extend the syntax of proof
scripts with subgoal variables. I then developed the notion of subgoal context.
I noticed that these subgoal variables were unusual in that they are checkable
expressions, and that the upshot of this is that the subgoal context is
understood as an output of the typing judgment.
The statements of all the theorems and their proofs were worked out by me.

Marcel Goh helped to develop the translation presented in
Sec.~\ref{sec:translation} and wrote some preliminary code for typechecking and
translating proof scripts. I was the primary implementor of the code, with some
help from Clare Jang who implemented the system of automatic tactics and the
code for overwriting proof scripts with their more refined forms.
Clare also was responsible for porting a few of the examples listed in
Sec.~\ref{sec:evaluation} and Marcel wrote a few of the smaller tests for
Harpoon.

The running example discussed in Chap.~\ref{chap:example} was designed by
Prof.~Pientka and me, and I owe the overall structure of that chapter to her.
I also owe it to her for how to properly motivate this work in
Chap.~\ref{chap:introduction}.

The literature review given in Chap.~\ref{chap:background}
and the comparative analysis of other systems in Sec.~\ref{sec:related-work} is
entirely my own work.

\section*{Funding}

The research conducted during this degree was funded by the Fonds de Recherche
du Qu\'ebec, Nature et Technologies (FRQNT) scholarship B1X \#258099.
I also want to recognize supplemental funding by my advisor, Brigitte Pientka.

%%% Local Variables:
%%% mode: latex
%%% TeX-master: "../main.tex"
%%% End:
