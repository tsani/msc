\begin{abstract}
% \begin{center}
%   \bfseries\textsc{Abstract}
% \end{center}

Beluga is a proof environment based on the logical framework LF that provides
infrastructural support for representing formal systems and proofs about them.
As a consequence, meta-theoretic proofs are precise and compact. However,
programmers write proofs as total recursive programs. This can be challenging
and cumbersome.

We present the design and implementation of Harpoon, an interactive proof
environment built on top of Beluga. Harpoon users develop proofs using a small,
fixed set of tactics. Behind the scenes, the execution of tactics elaborates a
proof script that reflects the subgoal structure of the proof. We model
incomplete proofs using contextual variables to represent holes.
We give a sound translation of proof scripts into Beluga programs which allows
us to execute them. Proof scripts and programs seamlessly interact and can be
used interchangeably.

We have used Harpoon for examples ranging from simple type safety proofs for
MiniML to normalization proofs including the recently proposed POPLMark Reloaded
challenge.
Our implementation is a part of Beluga on GitHub at
\url{https://github.com/Beluga-lang/Beluga} and the reference manual is
available at \url{https://beluga-lang.readthedocs.io/}.
\end{abstract}

\newpage

{
  \selectlanguage{french}
  \begin{abstract}
  \frenchspacing

Beluga est un assistant de preuves bas\'e sur le cadre logique (logical
framework) LF et donnant une infrastructure quant \`a la repr\'esentation de
syst\`emes formels et de preuves concernant ceux-ci.
Par cons\'equence, ces preuves metath\'eoriques sont pr\'ecises et compactes.
Cependant, on d\'eveloppe une preuve avec Beluga en tant que programme
r\'ecursif total, ce qui est difficile et encombrant.

Nous pr\'esentons alors la conception de Harpoon, une extension \`a Beluga, avec
lequel on fait le d\'eveloppement interactif de preuves.
L'utilisateur de Harpoon construit une preuve en utilisant un petit ensemble
ferm\'e \emph{d'actions}.
L'ex\'ecution de celles-ci \'elabore un \emph{texte de preuve} (proof script)
qui refl\`ete la structure des sous-objectifs qui se pr\'esentent
dans la preuve.
Nous mod\'elisons les preuves incompl\`etes en utilisant des variables
contextuelles pour repr\'esenter les sous-objectifs qui restent \`a r\'esoudre.
Nous \'etablissons de plus une mani\`ere de traduire les textes de preuve en
programmes Beluga traditionels, ce qui permet l'ex\'ecution de ces preuves.
Les textes de preuve et les programmes Beluga int\'eragissent facilement, donc
l'utilisateur peut choisir quel m\'ethode de preuve lui convient le mieux.

Nous avons \'evalu\'e Harpoon sur de nombreux exemples, en passant par des
th\'eor\`emes de pr\'eservation de typage et de progr\`es pour MiniML
ainsi que des th\'eor\`emes de normalisation tels que le r\'ecent d\'efi
POPLMark Reloaded.

Le code que nous avons \'ecrit fait maintenant partie de Beluga sur GitHub \`a
\url{https://github.com/Beluga-lang/Beluga} et le manuel de r\'ef\'erence est
disposible \`a \url{https://beluga-lang.readthedocs.io/}.
\end{abstract}
}

\vfill

%%% Local Variables:
%%% mode: latex
%%% TeX-master: "../main.tex"
%%% End:
