\chapter{Conclusion}

\section{Evaluation}

One should be able to use \Harpoon{} to prove anything that one could prove in
\Beluga. As mentioned in Sec.~\ref{sec:harpoon-actions}, we do not prove a
completeness theorem, so instead we have replicated a number of case studies
originally proven as functional programs in \Beluga.
(See table~\ref{table:evaluation}.)

\begin{table}[h]
  \centering
  \begin{tabular}{%
  m{0.4\textwidth} m{0.4\textwidth} %m{0.1\textwidth}
  }
  Case study
  & Main feature tested
  % & Remarks
  \\ \hline
  %%%%%%%%%%%%%%%%%%%%%%%%%%%
  % Examples from \textit{a framework for defining logics}
  % & Unboxing computational variables
  % & \multicolumn{1}{c |}{\---}
  % \\
  %%%%%%%%%%%%%%%%%%%%%%%%%%%
  % Length of context
  % & Splitting on context
  % & \multicolumn{1}{c |}{\ref{itm:evaluation-remark1}}
  % \\
  %%%%%%%%%%%%%%%%%%%%%%%%%%%%
  MiniML value soundness
  & Automatic solving of trivial goals
  % & \multicolumn{1}{c}{\ref{itm:evaluation-remark1}}
  \\
  %%%%%%%%%%%%%%%%%%%%%%%%%%%%
  MiniML compilation completeness
  & Unboxing program variables
  % & \multicolumn{1}{c}{\ref{itm:evaluation-remark3}}
  \\
  %%%%%%%%%%%%%%%%%%%%%%%%%%%%
  STLC type preservation
  & Automatic solving of trivial goals
  % & \multicolumn{1}{c}{\ref{itm:evaluation-remark5}}
  \\
  %%%%%%%%%%%%%%%%%%%%%%%%%%%%
  STLC type uniqueness
  & Open term manipulation
  % & \multicolumn{1}{c}{\ref{itm:evaluation-remark5}}
  \\
  %%%%%%%%%%%%%%%%%%%%%%%%%%%%
  STLC weak normalization
  & Advanced splitting
  % & \multicolumn{1}{c}{\ref{itm:evaluation-remark5}}
  \\
  %%%%%%%%%%%%%%%%%%%%%%%%%%%%
  STLC strong normalization \cite{POPLMarkReloaded:19}
  & Large development
  % & \multicolumn{1}{c}{\ref{itm:evaluation-remark5}}
  \\
  %%%%%%%%%%%%%%%%%%%%%%%%%%%%
  STLC alg. equality completeness \cite{Cave:MSCS18}
  & Large development
  % & \multicolumn{1}{c}{\ref{itm:evaluation-remark5}}
  \\
  %
\end{tabular}

  \caption{%
    Summary of proofs ported to \Harpoon{} from \Beluga.
  }
  \label{table:evaluation}
\end{table}

The first four examples are purely syntactic arguments that proceed by
straightforward induction.
The remaining examples involve more sophisticated features from \Beluga's
computation language such as inductive and stratified types used to encode
logical relations.

Recreating these case studies in \Harpoon{} was straightforward, since most
constructs from Beluga have clear analogues in \Harpoon.
One challenge was to decide when to use the \ttsufficess tactic, since most
reasoning in Beluga is purely forwards.

Porting these case studies provides us with insight as to future work regarding
automation.
In the syntactic case studies, proofs tend to proceed by case analysis on the
induction variable, inverting any other assumptions when possible, invoking
available induction hypotheses, and applying a few inference rules.
This general recipe could be automated in whole or in part to simplify the
development of similar simple proofs.

\section{Related work}

Although we devoted Chap.~\ref{chap:background} to discussing the prior work
related to Harpoon, we wish to now make a critical comparison between Harpoon
and some of that work.

The Hazelnut system is similar to \Harpoon{} in that its metatheory formally
describes partial programs and the user interactions that construct such a
program \cite{hazelnut}.
Whereas Hazelnut concentrates on programming, \Harpoon{} is an interface to
\Beluga, a proof assistant. Hazelnut's edit actions construct a simply-typed
program by successively filling holes, and types in Hazelnut may also contain
holes that are refined by edit actions.
In Hazelnut, there is a notion of cursor that is absent in Harpoon.
The focus of the Harpoon user is always on an (unsolved) subgoal, whereas in
Hazelnut, one can use certain actions to move to arbitrary locations in the
constructed expression.
Harpoon distinguishes clearly between expressions containing holes and
expressions free of holes, thanks to the subgoal context $\Omega$ that is part
of the typing judgments.
This context is also used to drive the selection of the next hole to work on
when considering an interactive session $\actionseq$.
The typing judgment for Hazelnut, in contrast, does not show the presence of
subgoals in the expression. A context analoguous to $\Omega$ is not necessary
as a means of finding the next subgoal to work on since the system is equipped
with general movement actions.

Abella is similar to the \Beluga{} project more broadly in that it is a
domain-specific language using HOAS for mechanizing metatheory
\cite{Gacek:IJCAR08, Gacek:JAR12}.
Its theoretical basis differs from \Beluga's, however, and it extends
first-order logic with a $\nabla$ quantifier to express properties about
variables. Contexts and simultaneous substitutions are expressed as inductive
definitions, but since they are not first-class one must separately establish
properties about them, regarding e.g. substitution composition and context
well-formedness. Interactive proof development in Abella follows the traditional
model: the proof state is manipulated using tactics drawn from a fixed set. No
proof object that witnesses the theorem is produced.

The VeriML system is grounded in Contextual Modal Type Theory, as is Beluga.
However, it differs in its goal, as it seeks to provide a very expressive
language for defining new tactics.
In our work, we restrict ourselves to a finite set of tactics with clearly
defined semantics, whereas in VeriML one can use effects such as state and
nontermination in order to build complex decision procedures in addition to more
basic tactics.
Both VeriML and Beluga follow a two-layer approach: whereas Beluga's object
language is contextual LF, VeriML's is \lhol, which the authors see as a common
core between Coq and HOL provers such as Isabelle and HOL-light.
The management of metatheoretic concerns such as substitutions and contexts is
low-level: one much explicitly model them, e.g. using lists.

Although much of the discussion in Chap.~\ref{chap:background} focused on tactic
languages for Coq, all of these differ from Harpoon in their aims.
Harpoon seeks simply to provide a form of user interaction together with a
well-specified semantics for those interactions based on an explicit notion of
partial program.
Ltac, on the other hand, is meant as a language for defining \emph{new} tactics,
and these tactics may implement decision procedures or restricted proof search
techniques.
Given Beluga's focus on mechanized metatheory, it is not clear yet that one
needs the ability to define decision procedures or use proof search techniques.
This is in fact the reason why we avoid using the word ``tactic'' to refer to
the interactive actions available in Harpoon: the notion of tactic is much
broader that what we aim for in Harpoon.
The MetaCoq system, in contrast with Ltac, is more foundational.
It seeks to provide a formalization of the Coq system within Coq itself.
Moreover, it appears more akin to a metaprogramming system than an interactivity
layer: one can write MetaCoq programs that are able to look up definitions and
inspect them, and whose effects generate new toplevel declarations.
Harpoon is humbler, in that it does not try to formalize Beluga within itself,
nor does it provide any way to generate new definitions.
Instead Harpoon restricts its actions to those that generate proof fragments
(terms).

% Consequently, it is difficult to obtain any assurance that a tactic one has
% developed is sound.
% Although user-definable tactics are a very powerful and attractive feature for a
% proof assistant, we prefer that \Harpoon{} use a fixed set of verified tactics
% until we have designed a suitable tactic development language that offers
% correctness guarantees about tactics defined in it.

% In contrast to having users directly write dependently-typed programs
% as in Agda or Twelf, proof assistants systems such as \Coq{}
% \cite{bertot/casteran:2004} extend the approach pioneered by
% \LCF \cite{edinburgh-lcf} by having users interactively develop a proof using
% \emph{tactics}. Behind the scenes, these are elaborated into a dependently-typed
% program.
% To enable replaying the proof on demand, a sequence of tactics is recorded as a proof script.
% Ideally, applying successfully a tactic to a proof state should
% only result in a new valid, consistent proof state, but this isn't always the case:
% user-defined tactics for \Coq{} constructed in the Ltac language
% \cite{Delahaye:LPAR00} are mostly unconstrained; it is \Coq's type-checker
% that verifies post hoc that the program generated by the tactics is
% valid. In addition, tactic languages often have other caveats: they are
% low-level (for example, we might refer to an assumption by its position in the
% list of assumptions), and the resulting proof scripts are often brittle and
% unreadable.
% %As we develop a proof interactively, automatically generated variable names for assumptions are introduced and we might refer to specific meta-variables in a goal based on the total number of meta-variables and based their position. For example, 1 identifies a meta-variable appearing in the current goal, where the last meta-variable appearing is assigned number 1, the second-last assigned number 2, and so on. As a consequence, such proof scripts can be very fragile to small changes in the specification.
% %
% An alternative to this approach is the declarative style. In this style, the
% subgoal structure of the proof is retained and the scope of local names is
% explicit. This proof style is most frequently associated today with the Isar
% language \cite{isar} for the Isabelle proof assistant \cite{isabelle} and is
% meant to improve human readability of proofs. It is however also more verbose to
% use. % and does not lend itself to just getting a job done.
% %
%
%
% Some related work to be discussed:\cite{Aboul-Hosn:MKM06}
%
% \paragraph{Twelf}
%
% \cite{Pfenning99cade,Schurmann98cade}
%
%
%
% \paragraph{Interactive Theorem Proving in Agda}
% Writing dependently typed programs can be challenging. In an effort to
% reduce the effort in constructing proof terms, dependently-typed
% programming language, \Agda, offers an interactive development
% environment, integrated via the text editor  \Emacs.
% %
% For example, in the \Emacs{} mode for \Agda, the user can issue a command to
% split on a variable. \Agda{} then loads the buffer, generates the appropriate
% case analysis syntax, sends it to \Emacs, and \Emacs{} splices this code back
% into the buffer. However, there is no theoretical foundation for the
% interaction between the interactive command-driven \Emacs{} mode and \Agda.
% When the user issues a command via the \Emacs{} interface, \Agda{}
% really executes this command on the internal syntax of
% the program, which is obtained from the external syntax in the user's buffer by
% a sophisticated type reconstruction algorithm. The code resulting from the
% command, in internal syntax, must then be translated to external syntax, so
% \Emacs{} can insert it into the buffer.  Although the metatheory of the
% elaboration from external to internal syntax is
% understood~\cite{elab-dep-copat-matching}, there is no metatheory for the
% reverse direction, from internal to external syntax and there is no
% type-theoretic foundation of refining a partial proof/program.  Therefore, it is
% \todo{check that is actually the case in Agda} not uncommon to have an
% interactive command generate invalid syntax or an ill-typed code fragment. The
% interactive \Emacs{} mode for \Beluga{}, similar to \Agda's, suffered from this
% same shortcoming.

\section{Final remarks}

In conclusion, we have presented \Harpoon, an interactive command-driven
front-end of \Beluga{} for mechanizing metatheoretic proofs.
Users develop proofs using interactive actions that elaborate a proof script
behind the scenes.
This elaboration's metatheory which we have presented shows that all
intermediate partial proofs are well-typed with respect to a context of
outstanding subgoals to resolve.
We have also developed the metatheory of proof scripts, giving a sound
translation to \Beluga{} programs.
This development relies crucially on reasoning about partial programs, which we
represent as containing contextual variables, called subgoal variables, that
capture the current typechecking state.
We have evaluated \Harpoon{} on a number of case studies, ranging from purely
syntactic arguments to logical relations.

In the future, we aim to improve the automation capabilities of \Harpoon.
At first, we wish to add a built-in form of proof search to assist in using the
\ttsolves command, perhaps replacing it entirely. In the long term, we hope to
apply insights gained from work on Cocon \cite{cocon} to enable users to define
custom tactics together with correctness guarantees about them.

%%% Local Variables:
%%% TeX-master: "../main.tex"
%%% End:
