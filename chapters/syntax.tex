\chapter{%
  A Logical Foundation for Interactive Theorem Proving%
}
\chaptermark{Logical Foundation}
\label{chap:theory}

In this chapter we give a logical foundation for interactive
command-driven theorem proving in \Beluga{} using Harpoon.
In particular, we describe Harpoon's interactive commands, their relationship
to proof scripts, and the translation from proof scripts into \Beluga{} programs.

% A key insight in this theoretical development is to use contextual variables to
% represent open subgoals in a proof.

%  describe the three languages with which our work concerns
% itself: %
% \Beluga{} (section~\ref{sec:beluga}), %
% proof scripts (section~\ref{sec:harpoon}), %
% and interactive sessions (section~\ref{sec:harpoon-actions}). %
% We also develop some elementary metatheory for interactive sessions, namely a
% soundness theorem showing that interactive sessions always produce well-typed
% proof scripts. In light of the translation described in
% section~\ref{sec:translation} from proof scripts to programs, this soundness
% theorem also establishes that interactive sessions always produce well-typed
% \Beluga{} programs.

\section{%
  Background:  proofs as programs%
}
\label{sec:beluga}

We have already described programming in Beluga in Sec.~\ref{sec:beluga-intro},
but that discussion remained at a very high level to give intuition about the
system.
Here we describe formally \Beluga's programming language where we can
describe (inductive) proofs as total recursive programs.
From a logical perspective, a \Beluga{} program witnesses a theorem in a
first-order logic equipped with induction over a specific index domain.
Recall that in general, there are many possible index domains to choose from.
Beluga chooses contextual LF together with first-class contexts as its index
language.
Keeping this in mind, we keep the index domain abstract in the
description of \Beluga{} below. We abstractly refer to terms and types
in the index language by \emph{index term} $C$ and \emph{index type}
$U$.
%
\[
  \begin{array}{l@{~}c@{~}r@{~}l@{~~~~~~}l@{~}c@{~}r@{~}l}
  \mbox{Index type} & U & ::= & \ldots  &
    \mbox{Index context} & {\Delta} & ::= & {\cdot \galt \Delta, X{:}U}\\
  \mbox{Index term} & {C} & ::= & {\ldots}    &
    \mbox{Index substitution} & {\theta} & ::= & {\cdot \galt \theta, C/X}
  \end{array}
\]
%
Variables occurring in index terms are declared in an index context
$\Delta$. We use index substitutions to model the runtime environment of index
variables. Looking up $X$ in the substitution $\theta$ returns the
index term $C$ to which $X$ is bound at runtime. The index context
$\Delta$ captures the information that is statically available and is
used during type checking.

In the examples from Chap.~\ref{chap:example}, the index domain included the
definitions for \lstinline!tp!, \lstinline!tm A!, \lstinline!step M M!, and
\lstinline!steps M M!.
Recall that to make statements about those index domain objects, we pair those
objects (and their types) together with the context in which they are meaningful.
In the grammar above, $U$ refers to such a contextual type and $C$ denotes a
contextual object, for example
\lstinline!( |- arr unit unit)! is the contextual type of
\lstinline!( |- lam \x. x)!.
Contextual objects may contain index
variables that are declared in $\Delta$. For example,
\lstinline!( |- steps M M)! is meaningful in the index context
$\Delta = $ \lstinline!A:( |- tp), M:( |- tm A)!.

We do not describe here the index language in full detail -- it is quite
technical, it has been described elsewhere, and such a level of detail is not
crucial for the understanding or our design of \Harpoon.
Instead, we refer the interested reader to
\cite{Thibodeau:ICFP16,JacobRao:stratified2018}.
\todo{These citations don't describe LF in full detail either...}
Instead we list several relevant properties of the index language to be
compatible with our current presentation.

\begin{description}
\item \emph{Type checking index terms.} $\Delta \proves C \chk U$
  % \item \emph{Decidability of Type Checking for Index Substitutions:} $\Delta \proves \theta \chk \Delta^\prime$
\item \emph{Substitution principle.} If $\Delta \proves \theta \chk \Delta'$ and $\Delta' \proves C \chk U$ then
  $\Delta \proves \msubst{\theta}C \chk \msubst{\theta}U$.
  % \item \emph{Decidable equality for index terms:} $\Delta \proves C_1 = C_2$.
  % \item \emph{Decidable unification of index terms:} $\Delta \proves C_1 \unif C_2$.
\item \emph{Coverage.}
  $\scovs (\Delta; U) = \vec{(C_k; \theta_k; \Delta_k; \Gamma_k)}$
  computes a covering set for $U$ in the meta-context $\Delta$
  such that for each $k$, the index pattern $C_k$ satisfies
  $\Delta_k \proves C_k \chk \msubst{\theta_k}U$.
  Moreover, it computes any well-founded recursive calls and includes them as
  part of $\Gamma_k$ (see \cite{Pientka:TLCA15}).
\end{description}


Below we describe the core fragment of \Beluga{}.
Terms are separated into those that we check against a given type and those for
which we synthesize a type.
To keep the presentation simple, we model (co)inductive and stratified types as
constants.
Types include simple functions $\tau_1 \to \tau_2$,
dependent functions $\Pi X{:}U.\,\tau$,
boxed types $[U]$, and constants $\const{b}~\vec{C}$ used to model (co)inductive
and stratified types.
Here $\const{b}$ stands for an indexed type family and recall $U$ stands for an
index type.
%
\[
  \begin{grammar}
    \category{Base Types}{\beta}{%
       \const{b}~\vec{C} \mid [U]
    }%
    \category{Types}{\tau}{%
      \beta \mid \Pi X{:}U.\,\tau \mid \tau_1 \to \tau_2
    }%
    \category{Checkable Terms}{e}{%
      \bar g \mid i \mid [C]
      \mid \tcases i\;\tofs \vec{p_k \bto e_k}
    }%
    \continue{%
      \tmlams X \bto e \mid \tfns x \bto e
    }%
    \continue{%
      \tlets {x = i}\;\tins e \mid \tletboxs {X = i}\;\tins e
    }%
    \category{Synthesizable Terms}{i}{%
      x \mid \const{c} \mid i~C \mid i~e \mid (e : \tau)
    }
    \category{Patterns}{p}{[C] \mid \const{c}~\vec{p} \mid x}
    \category{Context}{\Gamma}{%
      \cdot \mid \Gamma, x{:}\tau
    }
    \category{Subgoal context}{\Omega}{%
      \cdot
      \mid \Omega,\; g{:}\goal{}
      \mid \Omega,\;\bar g{:}\goal{}
    }
  \end{grammar}
\]
%
%%% Local Variables:
%%% TeX-master: "../main.tex"
%%% End:

%
Synthesizable terms include variables, constants, and simple and dependent
function eliminations.
All synthesizable terms are checkable.
Conversely, a type annotation allows embedding a checkable term as a
synthesizable term.
This notably enables using a contextual object as a \tcases scrutinee.

Checkable terms include simple and dependent function abstraction (\tfn{} and
\tmlam{} respectively), boxed index objects $[C]$, and a \tcases expression.
We also include for convenience two different let-expressions
$\tlets x = i\;\tins e$ and $\tletboxs X = i\;\tins e$,
although in principle each could be defined using a function (\tfns and \tmlams
respectively) that is immediately eliminated.
Such a translation must be type-directed in our bidirectional setting, as
functions are checkable terms which cannot appear on the left-hand side of a
function elimination.
Broadly speaking, such a translation would synthesize the type for $i$ in order
to construct a type annotation for the anonymous function to embed it as the
subject of the function elimination.

\begin{figure}[tp]
  \declarejudgment{%
  $\boxed{\belchkjudg{\Omega \mid \goalctx{}}{e}{\tau}}$
}{%
  \Beluga{} term $e$ checks against type $\tau$
}
%
\[
  \begin{array}{c}
    % PiBox introduction (mlam)
  \infer{%
    \belchkjudg{\Omega \mid \goalctx{}}{\tmlams X \bto e}{\Pi X{:}U.\, \tau}
  }{%
    \belchkjudg{\Omega \mid \Delta, X{:}U; \Gamma}{e}{\tau}
  }
  \quad
    % arrow introduction (fn)
  \infer{%
    \belchkjudg{\Omega \mid \goalctx{}}{\tfns x \bto e}{\tau_1 \to \tau_2}
  }{%
    \belchkjudg{\Omega \mid \Delta; \Gamma, x : \tau_1}{e}{\tau_2}
  }
    \\[1em]
    % switching from checking to synthesis
\infer{%
\belchkjudg{\Omega \mid \goalctx{}}{i}{\tau}}
{\belsynjudg{\Omega \mid \goalctx{}}{i}{\tau}}
    \quad
    % subgoal variable
    \infer{%
    \belchkjudg{g : (\goalctx{} \proves \tau) \mid \goalctx{}}{g}{\tau}
    }{}
\\[1em]
    % case expression
  \infer{%
    \belchkjudg{\Omega, \vec{\Omega_k} \mid \goalctx{}}{%
    \tcases i\;\tofs \vec{p_k \bto e_k}%
    }{\tau}
  }{%
    \begin{array}{c}
    \belsynjudg{\Omega \mid \goalctx{}}{i}{\beta}
      \quad
    \scovs (\Delta; \Gamma; \beta)
    = \vec{(p_k, \theta_k, \Delta_k, \Gamma_k)}
      \\
      \text{for all $k$. }
%        \belchkjudg{\goalctx{_k}}{p_k}{\msubst{\theta_k}\beta}
%      \quad
    \belchkjudg{\Omega_k \mid \goalctx{_k}}{e_k}{\msubst{\theta_k}{\tau}}
    \end{array}
  }

\\[1em]
    % ordinary let-expression
 \infer{%
    \belchkjudg{\Omega_1, \Omega_2 \mid \goalctx{}}{\tlets x = i\;\tins e}{\tau}
  }{%
    \belsynjudg{\Omega_1 \mid \goalctx{}}{i}{\tau^\prime}
    &
    \belchkjudg{\Omega_2 \mid \Delta; \Gamma, x : \tau^\prime}{e}{\tau}
  }
      \\[1em]
      % let-box expression
  \infer{%
    \belchkjudg{\Omega_1, \Omega_2 \mid \goalctx{}}{\tletboxs X = i \;\tins e}{\tau}
  }{
    \belsynjudg{\Omega_1 \mid \goalctx{}}{i}{[U]}
    &
    \belchkjudg{\Omega_2 \mid \Delta, X{:}U; \Gamma}{e}{\tau}
  }
\end{array}
\]

\declarejudgment{%
  $\boxed{\belsynjudg{\Omega \mid \goalctx{}}{i}{\tau}}$
}{%
  \Beluga{} term $i$ synthesizes type $\tau$
}
%
\[
  \begin{array}{c}
    % Look up a program variable in Gamma
\infer{
\belsynjudg{\cdot \mid \goalctx{}}{x}{\tau}
}{\Gamma(x) = \tau}
\quad
    % Look up a comp constant in the signature
    \infer{
    \belsynjudg{\cdot \mid \goalctx{}}{\const{c}}{\tau}
    }{%
    \text{Sig}(\const{c}) = \tau
    }
    \quad
    % Type annotation
    \infer{
    \belsynjudg{\Omega \mid \goalctx{}}{(e : \tau)}{\tau}
    }{
    \belchkjudg{\Omega \mid \goalctx{}}{e}{\tau}
    }
    \\[1em]
    % PiBox elimination
  \infer{%
    \belsynjudg{\Omega \mid \goalctx{}}{i\; C}{\msubst{C/X}{\tau}}
  }{%
    \belsynjudg{\Omega \mid \goalctx{}}{i}{\Pi X{:}U.\; \tau}
    &
      \Delta \proves C \chk U
  }
      \\[1em]
      % Arrow elimination
  \infer{
    \belsynjudg{\Omega_1, \Omega_2 \mid \goalctx{}}{i\; e}{\tau_2}
  }{%
    \belsynjudg{\Omega_1 \mid \goalctx{}}{i}{\tau_1 \to \tau_2}
    &
    \belchkjudg{\Omega_2 \mid \goalctx{}}{e}{\tau_1}
  }
  \end{array}
\]

\newcommand{\ctx}{~\textsf{ctx}}
\declarejudgment{%
  $\boxed{\vdash \Omega \ctx}$
}{%
 $\Omega$ is a valid subgoal context
}
\[
  \begin{array}{c}
\infer{\vdash \cdot \ctx}    {}
\qquad
\infer{\vdash (\Omega, g: (\goalctx{} \proves \tau)) \ctx}
{\vdash \Omega \ctx & \goalctx{} \vdash \tau : \textsf{type}}
  \end{array}
\]


%%% Local Variables:
%%% mode: latex
%%% TeX-master: "../main.tex"
%%% End:

  \caption{\Beluga's bidirectional type system, and well-formedness of subgoal contexts.}
  \label{fig:beluga-types}
\end{figure}

Last, the syntax of checkable expressions contains contextual variables $\bar g$
following \cite{Nanevski:ICML05,Boespflug:LFMTP11}, which we call
\emph{subgoal variables}.
A subgoal variable represents a typed hole in the program that remains to be
filled by the programmer.
It captures in its type $\goal{}$ the typechecking state at the point it occurs:
it remains to construct a term that checks against $\tau$ in the index context
$\Delta$ with the assumptions in $\Gamma$.
Observe that subgoal variables appear only in the \emph{term} language: this
ensures that subgoals cannot refer to each other.
Since subgoals are independent of each other, they may be solved in any order by
the user.
An expression is called \emph{complete} if it is free from subgoals, and
\emph{incomplete} otherwise.

Subgoal variables are collected in a \emph{subgoal context} $\Omega$.
Algorithmically, we understand a subgoal context $\Omega$ not as an input to the
typing judgments in Fig.~\ref{fig:beluga-types} but rather as an output: the set
of holes in the program is computed by the judgment.
This explains why we must \emph{check} a subgoal variable against a type $\tau$.

Most of the typing rules in Fig.~\ref{fig:beluga-types} are as expected.
To typecheck a \tcases expression, we infer the type of the expression that we
want to analyze, then generate a covering set consisting of the pattern and the
refinement substitution $\theta$.
We then verify that the given set of patterns matches the covering set using the
primitive $\scovs(\Delta;\Gamma;\beta)$ which in turn relies on the coverage
primitive $\scovs(\Delta;U)$ for index objects.
Similar to the coverage primitive for index types, the coverage primitive for
computation-level base types also generates well-founded recursive calls and
includes in $\Gamma_k$.
Concretely, $\Gamma_k$ is an extension of $\msubst{\theta_k}\Gamma$
that includes any program variables bound by the pattern as well as any
well-founded recursive calls.
Finally, we check each branch considering the index susbtitution $\theta_k$
that accounts for any index variable refinements induced by the pattern.
We omit the typing rules for patterns: given that patterns are a subset of
expressions, their typing mirrors expression typing.

As for the subgoal context $\Omega$, every rule's conclusion collects all
premise subgoal contexts to propagate the subgoals downwards.
Note that all subgoal variables are distinct and occur exactly once: we
allow neither weakening nor contraction for $\Omega$.
We distinguish between subgoal variables $\bar g$ that stand for programs from
those $g$ that stand for \emph{proofs} described in Sec.~\ref{sec:harpoon}.
We call these \emph{program} subgoal variables and \emph{proof} subgoal
variables, respectively.
Observe that the latter form of subgoal variable cannot appear within programs,
so the subgoal context $\Omega$ in Fig.~\ref{fig:beluga-types} only ever
contains program subgoal variables, of the form $\bar g$.
As expected, one can define a notion of substitution $\psubst{e^\prime/\bar g}e$
that eliminates a subgoal variable, satisfying the following theorem.

\begin{thm}[Subgoal Substitution Property 1]
  If $\belchkjudg{\Omega,\; \bar g{:}\goal{^\prime}\mid\goalctx{}}{e}{\tau}$
  and $\belchkjudg{\Omega^\prime\mid\goalctx{^\prime}}{e^\prime}{\tau^\prime}$,
  then $\belchkjudg{\Omega,\Omega^\prime\mid\goalctx{}}{\psubst{e^\prime/\bar g}}{\tau}$.
\end{thm}

\begin{proof}
  By induction on the first typing derivation.
\end{proof}

This property together with a corresponding one for proof subgoal variables is
central to the soundness of interactive proof development in
Sec.~\ref{sec:harpoon-actions}.

We omit the kinding rules and the well-formedness rules for
$\Delta$ and $\Gamma$. However, to emphasize that each subgoal type
cannot depend on other subgoal types, we include the well-formedness
rules for the subgoal context $\Omega$ in Fig.~\ref{fig:beluga-types}.

\section{\Harpoon{} Script Language}
\label{sec:harpoon}

To build proofs interactively, we introduce interactive commands,
called \emph{actions}, which are typed by the user into the \Harpoon{}
interactive prompt.
%
An action is executed on a particular subgoal and eliminates it
while possibly introducing new subgoals. Eliminating a subgoal with exactly one
new subgoal can be understood as transforming the initial subgoal.
%
\[
  \begin{grammar}
    \text{Actions} & \action & ::= &
    \ttintros \galt
    \ttsolves e \galt
    \ttbys i\;\ttass x \galt
    \ttsplits i \galt
    \ttsuffices\;i\;\ttbys \vec{k:\tau}
  \end{grammar}
\]
%
We only consider here a small subset of the tactics we support in
\Harpoon{}; others and new ones can be added following the same principles:
\ttintross introduces all available assumptions;
\ttsolve{} provides an explicit term to close the current subgoal;
\ttby{} allows to refer to a lemma, introduce an intermediate result, or use an
induction hypothesis, binding the result to a new program variable;
\ttunbox{} is the same as \ttby, but it binds its result to a new index variable;
\ttsplit{} generates a covering set of cases to consider for a given scrutinee;
\ttsuffices{} allows programmers to reason backwards via a lemma or a
constructor.

Behind the scenes, executing a tactic builds a \emph{proof script} that
represents it, and substitutes this script for the subgoal that the tactic
eliminates.
A proof script very closely reflects the structure of the proof, and the core
constructs of the proof script language closely resemble the syntax of actions.
%
\[
  \begin{grammar}
    \category{Proof Script}{P}{%
      g \galt
      D \galt
      \kwbys i\;\kwass x; P \galt
      \kwunboxs i\;\kwass X; P
    }
    \category{Directives}{D}{%
      \kwsolves e \galt
      \kwintross \hypoth{\goalctx{}}{P} \galt
      \kwsplits i\;\kwass \vec{\hypoth{\goalctx{_k}}{P_k}}
    }
    \continue{%
      \kwsufficesbys i\;\kwtoshows \vec{(k$\!\,$\mbox{\texttt{>}}~\tau_k \;\kwass P_k)}
    }
  \end{grammar}
\]

%%% Local Variables:
%%% TeX-master: "../main.tex"
%%% End:

%
We give the typing rules for proof scripts and directives in
Fig.~\ref{fig:harpoon-types}.
In its simplest form, a proof script $P$ is either a proof subgoal variable $g$
or a directive $D$ that describes how to prove a given goal.
The understanding of a subgoal variable here is the same as in the previous
section: it is a contextual variable of type $\goal{}$, representing that it
remains to show $\tau$ in the index context $\Delta$ with assumptions $\Gamma$.
As before, proof subgoal variables cannot depend on any other subgoal variables.
Given that expressions appear embedded within a proof script and that program
subgoal variables may appear within these expressions, we have that the subgoal
context $\Omega$ computed from a proof script may contain both forms of subgoal
variable.
We lift susbtitution of an expression for a program subgoal variable to proof
scripts, writing $\psubst{e/\bar g}{P}$, and we define substitution of a proof
script for a proof subgoal variable, written $\psubst{P^\prime/g}{P}$.
These forms of substitution admit a soundness property.

\begin{thm}[Subgoal Substitution Property 2] ~
  \begin{enumerate}
  \item
    If $\belchkjudg{\Omega^\prime\mid\goalctx{^\prime}}{e}{\tau^\prime}$
    and $\prfchkjudg{\Omega,\bar g{:}\goal{^\prime}\mid\goalctx{}}{P}{\tau}$,
    then
    $\prfchkjudg{%
      \Omega,\Omega^\prime\mid\goalctx{}%
    }{\psubst{e/\bar g}{P}}{\tau}$.

  \item
    If $\prfchkjudg{\Omega^\prime\mid\goalctx{^\prime}}{P^\prime}{\tau^\prime}$
    and $\prfchkjudg{\Omega,g{:}\goal{^\prime}\mid\goalctx{}}{P}{\tau}$,
    then
    $\prfchkjudg{%
      \Omega,\Omega^\prime\mid\goalctx{}%
    }{\psubst{P^\prime/g}{P}}{\tau}$.
  \end{enumerate}
\end{thm}
\begin{proof}
  1. By induction on the typing for $P$, using Subgoal Substitution Property~1
  when we arrive at an embedded expression $e$.
  2. By induction on the typing for $P$.
\end{proof}

\begin{figure}[tp]
  \declarejudgment{%
  $\boxed{\prfchkjudg{\Omega \mid \Delta; \Gamma}{P}{\tau}}$
}{%
  Proof script $P$ corresponds to theorem $\tau$
}
%
\[
  \infer{%
    \prfchkjudg{\Omega \mid \goalctx{}}{\kwbys i\;\kwass x; P}{\tau}
  }{%
    \belsynjudg{\cdot \mid \goalctx{}}{i}{\tau^\prime}
    &
    \prfchkjudg{\Omega \mid \Delta; \Gamma, x{:}\tau^\prime}{P}{\tau}
  }
\]
\[
  \infer{%
    \prfchkjudg{\Omega \mid\goalctx{}}{\kwunboxs i\;\kwass X; P}{\tau}
  }{%
    \belsynjudg{\cdot \mid \goalctx{}}{i}{[U]}
    &
    \prfchkjudg{\Omega \mid \Delta, X{:}U; \Gamma}{P}{\tau}
  }
\]
%
\[
  \infer{%
    \prfchkjudg{g : \goal{} \mid \goalctx{}}{g}{\tau}
  }{%
  }
  \quad
  \infer{%
    \prfchkjudg{\Omega \mid\goalctx{}}{D}{\tau}
  }{%
    \dirchkjudg{\Omega \mid\goalctx{}}{D}{\tau}
  }
\]

%%% Local Variables:
%%% mode: latex
%%% TeX-master: "../main.tex"
%%% End:
%
  \declarejudgment{%
  $\boxed{\dirchkjudg{\Omega \mid\goalctx{}}{D}{\tau}}$
}{%
  Directive $D$ establishes theorem $\tau$
}
%
\[
  \begin{array}{c}
  \infer{%
    \dirchkjudg{\cdot \mid\goalctx{}}{\kwsolves e}{\tau}
  }{%
    \belchkjudg{\cdot \mid \goalctx{}}{e}{\tau}
  }
    \\[1em]
  \infer{%
    \dirchkjudg{\Omega \mid\goalctx{}}{\kwintross \hypoth{\goalctx{^\prime}}{P}}{\tau}
  }{%
    \unrolljudg{\goalwith{\goalctx{^\prime}}{\beta}}{\goalctx{}}{e}{\tau}
    &
    \prfchkjudg{\Omega \mid\goalctx{^\prime}}{P}{\beta}
  }
\\[1em]
  \infer{%
    \dirchkjudg{\Union_k \Omega_k \mid\goalctx{}}{\kwsplits i\;\kwass\vec{\hypoth{\goalctx{_k}}{P_k}}}{\tau}
  }{%
    \begin{array}{c}
    \belsynjudg{\cdot \mid \goalctx{}}{i}{\beta}
      \quad
    \scovs (\Delta ; \Gamma \vdash \beta) = \vec{(\_; \theta_k; \Delta_k; \Gamma_k)}
      \\
    \text{for all $k$.}\;
    \prfchkjudg{\Omega_k \mid\goalctx{_k}}{P_k}{\msubst{\theta_k}{\tau}}
      \end{array}
  }
    \\[1em]
  \infer{%
    \dirchkjudg{\Union_k \Omega_k \mid\goalctx{}}{%
      \kwsufficesbys i\; \kwtoshows \vec{(k$\!\,$\mbox{\texttt{>}}~\tau_k \;\kwass P_k)}
    }
{\tau_0}
}{%
  \begin{array}{c}
    \belsynjudg{\cdot \mid \goalctx{}}{i}{%
      \Pi \Delta^\prime.~ \tau^\prime_n \to \ldots \to \tau^\prime_1 \to \tau^\prime_0
    }
    \\
    \Delta \proves ({\mathsf{id}}_\Delta,\theta) : (\Delta,\Delta^\prime)
    \quad
    \Delta \proves \msubst{\theta}\tau_0^\prime = \tau_0
    \\
    \text{for all $k \in [1,n]$}
    \quad
    \Delta \vdash \msubst{\theta}\tau_k^\prime = \tau_k
    \quad
    \prfchkjudg{\Omega_k \mid \goalctx{}}{P_k}{\tau_k}
  \end{array}
    }
  \end{array}
\]


%%% Local Variables:
%%% mode: latex
%%% TeX-master: "../main.tex"
%%% End:
%
  \caption{%
    The type system for \Harpoon{} proofs and directives%
  }%
  \label{fig:harpoon-types}%
\end{figure}

We extend a proof script using \kwby{} or \kwunbox{} to introduce new
assumptions.
The \kwby{} construct is used both for invoking a lemma, introducing
an intermediate result, and for appealing to an
induction hypothesis, extending $\Gamma$ with a new variable representing the
invocation.
The \kwunbox{} construct is identical to \kwby, but it binds a new
index variable, requiring that the term being unboxed synthesize a boxed
contextual type $[U]$.
Often, one unboxes an assumption from $\Gamma$, but it is convenient to allow
directly unboxing the result of an appeal to a lemma or an induction hypothesis,
without first requiring that this result be bound to a program variable in
$\Gamma$.

To check well-foundedness of appeals to induction hypotheses, we adopt the
approach used internally by \Beluga.
When the user states a theorem, they must explicitly give a termination measure.
Then, when splitting on a variable that participates in the termination measure,
we can generate all valid induction hypotheses in advance and store them in the
context $\Gamma$ (see \cite{Pientka:TLCA15}).
Finally, when we encounter an appeal to an induction hypothesis, it suffices to
check whether it is compatible with any of the precomputed ones in $\Gamma$.

There are four different directives we can use in a proof. The
simplest directive, $\kwsolves e$, merely ends a proof script by
giving a proof term $e$ as a witness of the appropriate type.
%
To introduce hypotheses into the index context $\Delta$ and the context
$\Gamma$, we use $\kwintross \hypoth{\goalctx{^\prime}}{P^\prime}$ where
$\goalctx{^\prime}$ are extensions of $\Delta$ and $\Gamma$.
%
The new goal type $\tau^\prime$ and the extended contexts $\goalctx{^\prime}$
are computed from the current subgoal by \emph{unrolling} it as in
Fig.~\ref{fig:unroll}.
This process stops once we reach a base type $\beta$.
As the type is unrolled, the judgment also constructs a partial program
that is used for the translation in Sec.~\ref{sec:translation}.
Understood algorithmically, the unrolling judgment encodes a total function
whose output respects an expected soundness property.
\begin{thm}[Soundness of Unrolling]
  \begin{enumerate}
  \item
    For any $\Delta$ and $\Gamma$ and any type $\tau$,
    there exists a unique term $e$ such that
    $\unrolljudg{(\goalctx{^\prime} \proves \beta)}{\goalctx{}}{e}{\tau}$, where
    $\Delta^\prime$ and $\Gamma^\prime$ are extensions of $\Delta$ and $\Gamma$,
    respectively.

  \item
    If $
    \unrolljudg{
      \goalwith{\goalctx{^\prime}}{\beta}
    }{\goalctx{}}{\tau}{e}
    $, then $
    \belchkjudg{
      g : \goalwith{\goalctx{^\prime}}{\beta} \mid \goalctx{}
    }{e}{\tau}
    $.
  \end{enumerate}
\end{thm}
%
\begin{proof}
  1. By induction on $\tau$; 2. by induction on the given derivation.
\end{proof}

\begin{figure}[htp]
    \centering
\declarejudgmentmulti{%
  $\boxed{%
    \unrolljudg{%
      \goalwith{\goalctx{^\prime}}{\beta}%
    }{\goalctx{}}{e}{\tau}%
  }$
}{
  \Beluga{} type $\tau$ unrolls to
  $\beta$ in the extended meta-context $\Delta^\prime$ and
  computation context $\Gamma^\prime$.
}
\newcommand{\ghole}{%
  \Delta^\prime ; \Gamma^\prime \mid \tau^\prime
}
%
\newcommand{\conc}[2]{%
  \unrolljudg{\goalwith {\goalctx{^\prime}} \beta}{\goalctx{}}{#1}{#2}
}
\newcommand{\prem}[2]{%
  \unrolljudg{\goalwith {\goalctx{^\prime}} \beta}{#1}{e}{#2}
}
%
\[
  \begin{array}{c}
  % Base case
  \infer{%
    \unrolljudg{\goalwith{\goalctx{}} \beta}{\goalctx{}}{g}{\beta}
  }{
%    \tau = [U]~\text{or}~\tau = \const{b}
  }
    \\[1em]
  %
  % Case for Arrow
  \infer{%
    \conc{\tfns x \bto e}{\tau_1 \to \tau_2}
  }{
    \prem{\Delta; \Gamma, x{:}\tau_1}{\tau_2}
  }
  \\[1em]
  %
  % Case for PiBox
  \infer{%
    \conc{\tmlams X \bto e}{\Pi X{:}U.\, \tau}
  }{
    \prem{\Delta, X{:}U; \Gamma}{\tau}
  }
  \end{array}
\]
%%% Local Variables:
%%% TeX-master: "../main.tex"
%%% End:
%
  \caption{Unrolling a \Beluga{} type.}%
  \label{fig:unroll}%
\end{figure}

% Unrolling a subgoal $\goal{}$ produces a new subgoal $\goal{^\prime}$ whose type
% $\tau^\prime$ is a base type and whose contexts are extensions of the input subgoal's.
% The unrolling also produces an incomplete expression $e$ such that
% $\belchkjudg{g : \goal{^\prime} \mid \goalctx{}}{e}{\tau}$, which is used
% crucially by the translation in Sec.~\ref{sec:translation}.

% The \kwsplit{} directive breaks up the proof into cases, by splitting on
% the term $i$.
%
% When using the \kwsplit{} directive, we usually split on a
% variable from $\Gamma$ or $\Delta$, but there is in general no necessity to
% restrict the set-up to case analysis of assumptions only. Instead we
% allow splitting on the type of any synthesizable expression: it
% can be more convenient for the user to directly consider the
% different cases that result from a recursive call (an appeal to an IH)
% or from appealing to a lemma. We support splitting on both index
% domain types and Beluga types.

The directive $\kwsplit$ breaks up the proof into cases, one for each
constructor of the type $\tau^\prime$ of the term $i$ being split on.
The \scov{} primitive computes a covering set of cases and generates
well-founded recursive calls based on the user-defined termination measure
(see \cite{Pientka:TLCA15}).
Each computed 4-tuple contains the pattern $p_k$ (unused here, but used and
explained in Sec.~~\ref{sec:translation}), a refinement substitution $\theta_k$
such that $\Delta_k \proves \theta_k : \Delta$, and contexts $\Delta_k$ and
$\Gamma_k$.
The proof is then decomposed into multiple branches, one for each $k$.
Each branch may introduce new assumptions, namely subderivations, and may refine
other assumptions via the substitution $\theta_k$.
It is also possible for \kwsplit{} to produce no cases, which corresponds to an
elimination principle for empty types.

Last, the \kwsuffices{} directive reasons backwards from by
current goal and introduces new proof obligations based on what it
would take to establish the current goal.
For simplicity, we only consider here types of the form
$\Pi \Delta^\prime.\tau_n^\prime \to \ldots \to \tau_1^\prime \to \tau_0^\prime$.
If the current goal type $\Delta \proves \tau_0$ is an instance of the target type
$\tau_0^\prime$, i.e. there exists a substitution $\theta$
s.t. $\Delta \proves \theta : \Delta^\prime$ and $\msubst{\theta}\tau_0
= \tau_0$, then the proof is complete if we can construct, for each $k$, a $P_k$
fullfilling the stated proof obligation $\msubst{\theta}\tau^\prime_k$. In practice, $\theta$ is
computed by unification given both the goal type $\tau_0$ and the
target type $\tau_0^\prime$.

\section{Interactive Proof Development}
\label{sec:harpoon-actions}

We now describe the generation of a proof script based on the
actions in Fig.~\ref{fig:interactive}. This relationship is by
design both immediate and straightforward.

%%%%% INTERACTIVE HARPOON TYPING
\begin{figure}
  \declarejudgment{%
  $\boxed{\itrnsjudg{\Delta;\Gamma}{\tau}{\action}{\Omega}{P}}$%
}{%
  Action $\action$ applied to subgoal $\goal{}$
  produces a proof script $P$ with subgoals $\Omega$%
}

\[
  \begin{array}{c}
  \infer{%
    \itrnsjudg{\Delta;\Gamma}{\tau}{\ttsolves e}{\Omega}{\kwsolves e}
  }{%
    \belsynjudg{\Omega\mid\goalctx{}}{e}{\tau}
  }
    \\[1em]
  \infer{%
    \itrnsjudg{\Delta;\Gamma}{\tau}{\ttintros}{g : \goalwith {\goalctx{^\prime}} \beta}{%
      \kwintross \hypoth{{\goalctx{^\prime}}}{g}
    }
  }{%
    \unrolljudg{\goalwith {\goalctx{^\prime}} \beta}{\goalctx{}}{e}{\tau}
  }
\\[1em]
  \renewcommand{\goal}[1]{(\Delta; \Gamma#1 \proves \tau)}
  \newcommand{\cmd}{\ttbys i\;\ttass x}
  \infer{%
    \itrnsjudg{\Delta;\Gamma}{\tau}{\ttbys i\;\ttass x}{\Omega, g : \goal{, x : \tau^\prime}}{%
      \kwbys i\; \kwass x; g
    }
  }{%
    \belsynjudg{\Omega \mid \Delta; \Gamma}{i}{\tau^\prime}
  }
\\[1em]
  \newcommand{\cmd}{\ttsplits i}
  \newcommand{\prf}{\kwsplits i\;\kwass\vec{\hypoth{\goalctx{_k}}{g_k}}}
  \newcommand{\gOmega}{g_k : (\goalctx{_k} \proves \msubst{\theta_k}{\tau})}
  \infer{%
    \itrnsjudg{\Delta;\Gamma}{\tau}{\cmd}{\Omega, \vec{\gOmega}}{\prf}
  }{%
    \belsynjudg{\Omega \mid \Delta; \Gamma}{i}{\beta}
    &
    \scovs (\goalctx{}; \beta) = \vec{(\_; \theta_k; \Delta_k; \Gamma_k)}
  }
\\[1em]
  \newcommand{\ang}{\mbox{\texttt{>}}~}%
  \newcommand{\kOmega}{g_k : (\goalctx{} \proves \tau_k)}%
  \infer{%
    \begin{array}{c}
    \itrnsjudg{\Delta ; \Gamma}{\tau_0}{\ttsuffices~i~\ttbys \vec{\tau_k}}
    {\\[0.5em] \Omega, \vec{\kOmega}}{\kwsufficesbys i~\kwtoshows \vec{k\ang g_k}}
    \end{array}
  }{
    \begin{array}{c}
    \belsynjudg{\Omega \mid \goalctx{}}{i}{\Pi \Delta^\prime.\tau^\prime_n \to \ldots \to
    \tau^\prime_1 \to \tau_0^\prime}
      \\[0.5em]
      \Delta \vdash \theta :  (\Delta,\Delta^\prime)
      \quad
      \Delta \vdash \msubst{\theta}{\tau_k^\prime} = \tau_k
      \quad
      \Delta \vdash \msubst{\theta}{\tau^\prime_0} = \tau_0
    \end{array}
}
  \end{array}
\]


%%% Local Variables:
%%% mode: latex
%%% TeX-master: "../main.tex"
%%% End:
%
  \caption{%
    Typing of interactive actions and elaboration into proof scripts.
  }%
  \label{fig:interactive}%
\end{figure}

Each action is simply elaborated into its corresponding construct in the proof
script language, using subgoal variables where appropriate to explicitly model
the remaining proof obligations.
%
% The \ttsolve{} action is the simplest: it merely solves the current goal when
% given a term $e$ of the current goal type. Often, this term is simply a
% variable, e.g. obtained by induction hypothesis, or an inference rule applied to
% a variable.
% % Our implementation performs some rudimentary automation to detect
% % available assumptions that match the current goal type. This already eliminates
% % certain trivial subgoals from proofs.
%
% For \ttintros{}, \ttby{}, and \ttunbox, the current  subgoal is replaced by a new
% subgoal within an extended context of assumptions.
% %
% % Our implementation automatically invokes \ttintros{} when the goal
% % is a function type.
%
% The action $\ttsplits i$ relies on the auxiliary function \scov{} to
% compute a covering set of cases for the term $i$ (see
% \cite{Pientka:TLCA15}). For empty types, this set may be empty
% resulting in zero splits and the subgoal is merely eliminated.
%
% Last, the action $\ttsuffices~i~\ttby~\vec{\tau_k}$ supports backwards
% reasoning provided that the target type of $i$ is an instance of the
% overall goal type $\tau$.
%
Multiple actions can be sequenced to form an interactive
\emph{session} $\actionseq$. A session is an idealized representation of how the
user interacts with the proof assistant.
%
\[
  \begin{grammar}
    \category{Session}{\actionseq}{\cdot \mid \action, \actionseq}
  \end{grammar}
\]
%

The proof script $P$ that results from a session $\actionseq$ is well-typed.
To see this, we first establish that the partial proof script generated by a
single action indeed solves the subgoal in which it is executed.

\begin{figure}[ht]
  \declarejudgment{%
  $\boxed{\itranssjudg{\Omega}{P}{\actionseq}{\Omega^\prime}{P^\prime}}$%
}{%
  Sequence of actions $\actionseq$ transforms proof script $P$ with
  subgoals $\Omega$ into proof script $P^\prime$ with subgoals 
  $\Omega^\prime$.
}


\[
  \begin{array}{c}
  \infer[\irefl]{%
    \itranssjudg{\Omega}{P}{\cdot}{\Omega}{P}
  }{}
\\[0.5em]
  \infer[\isingle]{%
    \itranssjudg{\Omega_1, g : \goal{}}{P}{\action, \actionseq}{\Omega_3}{Q}
  }{%
    \itrnsjudg{\Delta;\Gamma}{\tau}{\action}{\Omega_2}{P^\prime}
    &
    \itranssjudg{\Omega_1, \Omega_2}{\psubst{P^\prime/g}{P}}{\actionseq}{\Omega_3}{Q}
  }
    
  \end{array}
\]
%
  \caption{%
    Rules for sequencing interactive \Harpoon{} actions.
    Note that we can reorder $\Omega$ which allows us in principle to work on
    any subgoal in $\Omega$ in the \isingle{} rule.%
  }%
  \label{fig:harpoon-session}
\end{figure}
%

%Recall that we begin with proving the theorem statement which
%corresponds to a \Beluga{} type $\tau$. The initial statement is
%closed and based on it we form the
%\emph{initial subgoal} context $g : \goalclosed$ and the initial proof
%script $g$. %
%%
%As we develop the proof, we introduce new subgoals nested inside the
%proof script.

% We say that a session $\actionseq$ is \emph{complete} if it ends in a proof
% script $P^\prime$ containing no subgoals, i.e.
% \[
%   \itranssjudg{\Omega}{P}{\actionseq}{\cdot}{P^\prime}
% \]

\begin{thm}[Action Soundness]
  \label{thm:interactive-command-correctness}
  \item If $\itrnsjudg{\Delta; \Gamma}{\tau}{\action}{\Omega}{P}$,
  then $\prfchkjudg{\Omega \mid \Delta ; \Gamma}{P}{\tau}$.
\end{thm}
\begin{proof}
  By case analysis on the given derivation.
\end{proof}

Then, given that individual actions produce well-typed proof fragments, it
suffices to use the fact that subgoal substitution preserves types to establish
that a whole session $\actionseq$ generates a well-typed proof script.

\begin{thm}[Session Soundness]
  \label{thm:interactive-tps}
\item
  If $\prfchkjudg{\Omega \!\mid\!\goalctx{}}{\!P}{\tau}$
  and $\itranssjudg{\Omega}{\!P}{\actionseq}{\Omega^\prime}{\!P^\prime}$,
  then $\prfchkjudg{\Omega^\prime\! \mid\! \goalctx{}}{\!P^\prime}{\tau}$.
\end{thm}
\begin{proof}
  By induction using the previous theorem and the subgoal substitution property.
\end{proof}

What one would like to establish next is a completeness result: any
\emph{provable} statement $\tau$ admits a session $\actionseq$ whose elaboration
generates a proof script $P$ such that $\prfchkjudg{\cdot\mid\goalctx{}}{P}{\tau}$.
As a proxy for provability and in light of the translation presented in the
following section, one can consider statements $\tau$ having a Beluga term $e$
that check against that type.
Sadly, we cannot establish completeness for now, since Beluga supports deep
pattern matching (in which a pattern consists of nested constructors), whereas
the \kwsplits directive can only capture 1-deep patterns.
\inlinetodo{Prove a restricted completeness theorem, in which patterns are nice?}

\section{Translation}
\label{sec:translation}

The translation in Fig.~\ref{fig:translation} from proofs to \Beluga{} programs
is now straightforward. An \kwunbox{} becomes a \tletboxs
construct in \Beluga. Similarly, $\kwbys i\;\kwass x$ translates into
a \tlet-expression. The translation of directives is also direct. For
\kwintros, we already built an incomplete expression $e$ when we were unrolling
the type $\tau$, so it suffices to translate $P$ to an expression $e^\prime$ and
perform a substitution. The soundness of unrolling and the subgoal substitution
property ensure that this preserves types.
The \kwsplit{} directive translates to
a case-expression in \Beluga{}, making use of the patterns produced by \scov.
Finally the \kwsuffices{} directive translates into a function application.
The following soundness property follows immediately.

% TRANSLATION RULES
\begin{figure}[htp]
  \declarejudgmentmulti{%
  $\boxed{\prftransjudg{\Omega \mid \goalctx{}}{P}{e}{\tau}}$
}{%
  Proof $P$ is translates to \Beluga{} term $e$
}
%
\[
  \begin{array}{c}
  % switch from proof to directive
  \infer{%
    \prftransjudg{\Omega \mid \goalctx{}}{D}{e}{\tau}
  }{%
    \dirtransjudg{\Omega \mid \goalctx{}}{D}{e}{\tau}
  }
  \quad
  % subgoal variable
  \infer{%
    \prftransjudg{g : \goal{} \mid \goalctx{}}{g}{g}{\tau}
  }{}
    \\[1em]
  % by command
  \infer{%
    \prftransjudg{\Omega \mid \goalctx{}}{\kwbys i\;\kwass x; P}{\tlets x = i\;\tins e}{\tau}
  }{%
    \belsynjudg{\cdot \mid \goalctx{}}{i}{\tau^\prime}
    &
    \prftransjudg{\Omega \mid \Delta; \Gamma, x : \tau^\prime}{P}{e}{\tau}
  }
      \\[1em]
  % unbox command
  \infer{%
    \prftransjudg{\goalctx{}}{\kwunboxs i\;\kwass X; P}{%
      \tletboxs X = i\;\tins e
    }{\tau}
  }{%
    \belsynjudg{\cdot \mid\goalctx{}}{i}{[U]}
    &
    \prftransjudg{\Omega \mid \Delta, X{:}U; \Gamma}{P}{e}{\tau}
  }
  \end{array}
\]

~

\declarejudgment{%
  $\boxed{\dirtransjudg{\goalctx{}}{D}{e}{\tau}}$
}{%
  Directive $D$ translates to \Beluga{} term $e$
}

\[
  \begin{array}{c}
  % solve command
  \infer{%
    \dirtransjudg{\cdot \mid \goalctx{}}{\kwsolves e}{e}{\tau}
  }{\belchkjudg{\cdot \mid \goalctx }{e}\tau}
    \\[1em]
  \infer{%
    \dirtransjudg{\Omega \mid \goalctx{}}{%
      \kwintross \hypoth{\goalctx{^\prime}}{P}%
    }{\psubst{e^\prime/g}{e}}{\tau}
  }{%
    \unrolljudg{\goal{^\prime}}{\goalctx{}}{e}{\tau}
    &
    \prftransjudg{\Omega \mid \goalctx{^\prime}}{P}{e^\prime}{\tau^\prime}
  }
    \\[1em]
  \infer{%
    \dirtransjudg{\Union_k \Omega_k \mid \goalctx{}}{%
      \kwsplits i\;\kwass \vec{\hypoth{\goalctx{_k}}{P_k}}
    }{%
      \tcases i\;\tofs \vec{p_k \bto e_k}
    }{\tau}
  }{%
    \begin{array}{l}
    \belsynjudg{\cdot\mid\goalctx{}}{i}{\beta}
      \quad
    \scovs (\Delta; \Gamma; \beta) =
    \vec{(p_k; \theta_k; \Delta_k; \Gamma_k)}
      \\
    \text{for all $k$. }
    \prftransjudg{\Omega_k \mid \goalctx{_k}}{P_k}{e_k}{%
      \msubst{\theta_k}{\tau}%
    }
    \end{array}
  }
      \\[1em]
  % suffices command
  \infer{%
    \dirtransjudg{\Union_k \Omega_k \mid \goalctx{}}{%
      \kwsuffices\;i\;\kwbys \vec{\tau_k \;\kwass P_k}
    }{i\;\vec{C_j}\;\vec{e_k}}{\tau_0}}
   {%
     \begin{array}{l}
       \belsynjudg{\cdot \mid \goalctx{}}{i}{%
         \Pi \Delta^\prime.\,
         \tau_n^\prime \to \ldots \to \tau_1^\prime \to \tau_0^\prime%
       }
       \quad
       \Delta \vdash \theta : \Delta, \Delta^\prime
       \\
       \Delta \vdash \msubst{\theta}{\tau_k^\prime} = \tau_k
       \quad
       \dirtransjudg{\Omega_k \mid \goalctx{}}{P_k}{e_k}{\tau_k}
     \end{array}
  }
    \end{array}
\]%
\mbox{where} $\theta = C_1/X_1, \ldots C_m/X_m$%
%
%%% Local Variables:
%%% mode: latex
%%% TeX-master: "../main.tex"
%%% End:
%
  \caption{%
    The translation from a \Harpoon{} proof script to a \Beluga{} program.%
  }%
  \label{fig:translation}
\end{figure}%

\begin{thm}[Translation Soundness]
  \mutualtheorems
  \begin{enumerate}
  \item
    For any proof script $P$, if $\prfchkjudg{\Omega \mid \goalctx{}}{P}{\tau}$,
    then there exists a unique expression $e$ such that
    $ \prftransjudg{\Omega \mid \goalctx{}}{P}{e}{\tau} $.

  \item
    For any directive $D$, if $\dirchkjudg{\Omega\mid\goalctx{}}{D}{\tau}$,
    then there exists a unique expression $e$ such that
    $ \dirtransjudg{\Omega\mid\goalctx{}}{D}{e}{\tau} $.
  \end{enumerate}

  Moreover, in both cases $e$ is such that
  $\belchkjudg{\Omega\mid\goalctx{}}{e}{\tau}$.
\end{thm}

\begin{proof}
  By induction on the translation derivation.
\end{proof}

%
%%% Local Variables:
%%% TeX-master: "../main.tex"
%%% End:
