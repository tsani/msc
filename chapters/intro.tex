\chapter{Introduction}
\label{chap:introduction}

Mechanizing metatheory about formal systems such as programming languages and
logics plays an important role in establishing trust in formal developments. One
key question in this endeavour is how to represent variables, (simultaneous)
substitution, assumptions, derivations that depend on assumptions, and the proof
state, as our choices may impact how easy or how cumbersome it will be to
develop proofs about formal systems.

\Beluga~\cite{Pientka:IJCAR10,Pientka:CADE15} is a proof environment which
provides sophisticated infrastructure for implementing formal systems based on
the logical framework LF~\cite{Harper93jacm}.
This allows programmers to uniformly specify syntax, inference rules, and
derivation trees using higher-order abstract syntax (HOAS) and relieves users
from having to build custom-support to manage variable binding,
renaming, and substitution.
%
Following the Curry-Howard correspondence, \Beluga{} users develop inductive
metatheoretic proofs about formal systems by writing a total recursive
dependently-typed program by pattern matching on derivation trees. Proof
checking then amounts to type checking the user's program.
\Beluga{} hence follows in the foot steps of proof checkers such as Automath
\cite{Nederpelt:94}, Agda \cite{Norell:phd07}, and specifically Twelf \cite{Pfenning99cade}.

%
%While developing proofs in a proof assistant is hard, writing a dependently typed program that corresponds to the proof is even harder.
While writing a proof as a dependently typed program is a beautiful idea, it also
can be challenging and cumbersome. This limits the wide spread use of
dependently-typed programming languages for mechanizing proofs in
general. Hence, many proof assistants in this domain provide some form
of interaction: for example, Agda \cite{Norell:phd07} supports leaving holes
(questionmarks) and writing partial programs which can later be
refined using a fixed limited set of interactions. However a clear
specification and theoretical foundation of how these interactions
transform programs is largely missing. In \Coq{}
\cite{bertot/casteran:2004} users interactively develop a proof using
\emph{tactics}. Behind the scenes, a sequence of tactic applications
is elaborated into a dependently-typed
program. Ideally, applying successfully a tactic to a proof state should
only result in a new valid, consistent proof state, but this isn't always the case:
user-defined tactics for \Coq{} constructed in the Ltac language
\cite{Delahaye:LPAR00} are mostly unconstrained; it is \Coq's typechecker
that verifies post hoc that the program generated by the tactics is
valid. A common additional caveat of tactic languages
is that often, the resulting proof script is brittle and
unreadable.
% they are low-level (for example, we might refer to an assumption by its position in the list of assumptions),
% From a user's perspective programs become the by-product
% of an interactive proof session.


% How can we support interactive development of proofs and at the same time keep a tight connection between the proof development on the one hand the  program development on the other? --
%As a user, we clearly want to build at least part of the proof (program) interactively.



% our goal: design a a tactic-driven interactive theorem proving interface that retains the subgoal structure and is hence less brittle than existing approaches.

In this paper, we present the design and implementation of \Harpoon, an
interactive proof environment built on top of \Beluga, where programmers develop
proofs by a fixed set of tactics.
The user invokes a tactic in the context of a subgoal in order to transform,
split, or solve it.
Our fixed set of tactics is largely inspired by similar systems such as Twelf
\cite{Schurmann98cade} or Abella \cite{Gacek:IJCAR08} supporting introduction of
assumptions, case-analysis, and inductive reasoning, as well as both forward and
backward reasoning styles.
As \Harpoon{} is built on top of \Beluga{}, its tactics can also refer to a
Beluga{} programs to provide an explicit proof witness to justify a proof step.
The ability to seamlessly mix programming with command-driven interactive
theorem proving is particularly useful when appealing to a lemma and switching
between proving and programming.
Finally, successful tactic application is guaranteed to transform a valid proof
state into another valid proof state.
%
\Harpoon's command-driven front-end generates automatically as a result a proof
script that retains the subgoal structure. We think of a proof script as an
intermediate proof representation language to facilitate translation to
other formats, such as into (executable) \Beluga{} programs as shown in this paper or
perhaps eventually into a human-readable proof format.
Our specific contributions are the following:

%In this paper, we show the elaboration of structured proof scripts into well-typed \Beluga{} programs.
% Hence, our whole proof development pipeline in \Harpoon{} is fully specified and at every given point we can type check our intermediate result.
%
% %
% \Harpoon{} is built as an extension to \Beluga{}, and we allow mixing proof
% scripts and programs within a single file [but not witihin a program?], so proofs may depend on programs and vice versa. This enables users to select the approach  that best suits the theorem they are proving.
%
%
% We demonstrate in section~\ref{sec:example} the use of \Harpoon{} to prove a
% simple theorem, and explain how the interactive actions used are elaborated into
% the final proof script.
%
% %
% In section~\ref{sec:syntax}, we develop the metatheory of interactive \Harpoon{}
% sessions and of \Harpoon{} proof scripts. Notably, we prove that an interactive
% session always produces well-typed proof scripts.
% %
% Finally, we develop in section~\ref{sec:translation} a translation from
% \Harpoon{} proofs to \Beluga{} programs and we prove the translation to be
% type-preserving. This guarantees that proof scripts indeed are
% \emph{proofs}. Together with the soundness result from section~\ref{sec:syntax},
% this allows us to view the interactive environment as a structured editor for
% \Beluga{} programs that is guaranteed to produce only well-typed programs.
%

\begin{figure}
  \centering
  \begin{tikzpicture}

 \node[] (Box0) {
   \begin{tabular}{p{2cm}}
Specification \\[0.5em] \textbf{Statement}
   \end{tabular}
};

 \node[node distance=2.6cm, right= of Box0, block] (Box1) {
   \begin{tabular}{p{4cm}}
     Specification \\[0.5em] \textbf{Statement}
   \end{tabular}
   \nodepart{two} Proof  script};

 \node[node distance=2cm, right= of Box1, block] (Box2) {
   \begin{tabular}{p{4cm}}
     Specification \\[0.5em] \textbf{Statement}
   \end{tabular}
   \nodepart{two} Program};

 \path[pil,draw] (Box0) edge node[above] (label) {action} (Box1);
 \path[pil,draw] (Box1) -- (Box2);

\end{tikzpicture}

  \caption{\Harpoon{} Design Overview}
  \label{fig:harpoon}
\end{figure}

\begin{itemize}
\item
  We present the design and implementation of \Harpoon, an
  interactive command-driven front-end of \Beluga{} for mechanizing
  meta-theoretic proofs.
  Starting from a user-specified theory (including both its syntax and its
  judgments), users interactively develop meta-theoretic proofs using tactics.
  In tutorial style, we develop a short proof in Sec.~\ref{sec:example} by way
  of giving a whirlwind tour of the main supported tactics in \Harpoon.

  % and a meta-theoretic statement one wants to prove, users develop the proof
  % interactively using proof actions (see
  % Fig.~\ref{fig:harpoon}).
  % These actions are high-level and focus on
  % how we develop meta-theoretic proofs and include case analysis,
  % applying a lemma or an induction hypothesis, or providing a concrete
  % proof witness for a subgoal. Notably, uses of substitution and weakening
  % properties are inherited from LF and are used silently. The
  % result will be a structured proof script which subsequently can be
  % translated into a \Beluga{} program. While it is in principle
  % possible to directly generate a \Beluga{} program, proof scripts
  % help to characterize more clearly where open subgoals can occur
  % and what conditions they must satisfy. They also allows us to
  % naturally combine forward and backward reasoning.

\item
  We define in Sec.~\ref{sec:harpoon} a proof script language that reflects
  the proof structure laid out by the user and clearly separates forwards and
  backwards reasoning.
  We describe formally the relation between interactive tactics and proof
  scripts in Sec.~\ref{sec:harpoon-actions} and prove soundness of interactive
  proof construction.
  We give a type-preserving translation from proof scripts to \Beluga{} programs
  in Sec.~\ref{sec:translation}.
  This guarantees that proof scripts actually represent proofs and allows proof
  scripts to be not only typechecked, but also executed.
  Fig.~\ref{fig:harpoon} summarizes the connection among tactics, proof
  scripts, and \Beluga{} programs.
  %%% I think we should avoid *arguing* why we make the choices we make in this
  %%% list, and instead concentrate on presenting the contributions.
  % Although it is in principle possible to directly generate a \Beluga{} program
  % as a result of an interactive session, the proof script language helps
  % characterize what positions subgoals may appear in and what conditions they
  % must satisfy.
  % Further, we expect it to be helpful in a later translation to human-readable
  % proofs.

\item
  We characterize and reason about incomplete programs using contextual
  types.
  A variable of such a type represents a hole in the proof, i.e. a statement to
  prove together with a set of available assumptions.
  Our formalism of incomplete proofs is such that holes are independent of each
  other and may be solved in any order.
  We show that incremental proof development amounts of successively applying
  contextual substitutions to eliminate holes, while possibly introducing new
  ones.
  % \Harpoon{} guarantees that a successful action application
  % transforms a valid proof state into another valid proof state.
  % This crucially relies on characterizing and reasoning about holes in
  % incomplete proofs and programs.
  % We exploit the power of contextual types to describe holes in incomplete
  % proofs together with the context(s) in which the hole occurs.
  % Most notably, the contextual subgoal type cannot refer to other open subgoals
  % (or holes).
  % This allows us to reason modularly about incremental proof development via
  % actions and at the same time constrains the actions one can
  % take.
  % Interactively developing a proof amounts to successively applying contextual
  % substitutions to eliminate holes.

\item
  \Harpoon{} is implemented as part of \Beluga{} and is available
  at: \meh{???}.
  We have used it for a range of representive examples from the
  \Beluga{} library, in particular type safety proofs for
  MiniML, normalization proofs for the simply-typed lambda calculus
  \cite{Cave:MSCS18},
  % a higher-order solution to POPLMark \cite{Pientka:TPHOLs07},
  benchmarks for reasoning about binders
  \cite{Felty:MSCS17,Felty:orbi-survey}, and the recent POPLMark Reloaded
  challenge \cite{POPLMarkReloaded:19}.
  These examples cover a wide range of aspects that arise in the
  proof development such as complex reasoning with and about contexts,
  context schemas, and substitutions.
\end{itemize}

% While we design an interactive proof development engine for \Beluga, we believe there are some general lessons one can draw from our design:

% \paragraph{Remark [possibly to be moved elsewhere or to be removed -bp]} One might ask, why not generate directly from the interactive commands a dependently-typed program. There are several reasons why we chose to introduce a proof script language as an intermediate language: 1) When proving a property, we focus on the logical statement (= type) and the logical rules of inference. While there is an isomorphism between proofs and programs, we still think that it is often easier to think of the proof rather than the program. As our language allows us to embed programs into proof scripts, both can live together and the user can mix them and ultimately choose what is most convenient. 2) When writing a dependently typed program, one needs to understand the proof/program state for holes (open goals). Keeping track of types in the programmers head works for simply typed programs, but for dependently typed programs this becomes increasingly challenging. Hence, systems such as \Agda{} (or \Beluga) display the proof state for a given open subgoal (hole). This gives some indication of how important thinking about types rather than programs becomes.   3) Refining dependently typed programs is challenging in practice -- this is mostly due to the fact that a programer omits implicit arguments in the source program to make dependently typed programs more readable and compact; the program is then elaborated into an internal abstract syntax tree where all the omitted information has been inferred. When refining a partial dependently typed program, we typically refine the internal abstract syntax tree -- however to return the result of the program refinement back to the user, we need to again erase implicit arguments.

% A proof script which is structured representation of the proof attempt allows programmers to seamlessly switch between programming and proving; as proof scripts are more verbose, we can state clearly what it means to refine a proof script; as the proof script is generated automatically and not written by the programmer, it does not matter that it is verbose; verbosity in fact helps to generate programs from proof scripts; we also see proof scripts as a good intermediate representation to generate human-readable proofs.

%%% Local Variables:
%%% mode: latex
%%% TeX-master: "../main.tex"
%%% End:
