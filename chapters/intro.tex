\chapter{Introduction}
\label{chap:introduction}

Properties upheld by a formal system such as a logic or a programming language
are called its \emph{metatheory}.
Developing the metatheory of formal systems such as logics and programming
languages is central to establishing trust in those systems.
Such trust is essential since quite sophisticated programming languages, having
huge numbers of features, are used to create the vast bodies of software that
surround us in the modern world.
For programmers to have any hope of creating software free of bugs, they must
rely on guarantees made to them by the programming language they use.
If that language makes safety promises that turn out to be broken, then code
believed be bug-free may in fact contain bugs.
Even worse, these bugs may lead to security vulnerabilities in applications.
\inlinetodo{%
  Find a citation for a security vulnerability known to originate from a
compiler or interpreter bug?%
}%

Sadly, as a programming language becomes more complex, so does its metatheory.
Developing a proof on paper, although an excellent strategy for gaining an
intuition for the proof's correctness, does not provide bulletproof assurance of
its correctness.
Instead, one can seek a higher standard of correctness by \emph{mechanizing} the
metatheory.
That is, one can verify a proof using software called a proof assistant.

Alas, using a proof assistant is not as simple as one might hope.
One must first \emph{encode} the system of study in the assistant.
Next, one must formulate the statements of the theorems and their proofs in a
way that the assistant can automatically verify.
How one goes about resolving both of these concerns depends greatly on the
chosen proof assistant.

Proof assistants come in many shapes and flavours, as building such an
assistant requires that one make some extremely impactful choices.
What underlying theory will it use?
How will a user interact with it?
What sorts of proofs is it specialized for, if any?
When mechanizing metatheory, a key question is:
how to represent variables, (simultaneous) substitutions, assumptions,
derivations that depend on assumptions, and the proof state?
These decisions greatly influence how easy (or how cumbersome) it might be to
encode various formal systems and to reason about them.

% That is, these assistants exploit a beautiful phenomenon that links proofs with
% programs, called the Curry-Howard correspondence.
%
% This correspondence first observes that the connectives from logic, such as
% implications and quantifiers, can be reinterpreted as type formers in an
% appropriate functional programming language.
% For example, a logical proposition ``$A$ implies $B$'', written formally as
% $A \implies B$, can be reinterpreted as the type $A \to B$, i.e. the type of
% functions from type $A$ to type $B$.
% This is called the \emph{propositions-as-types} principle.
% Second, the correspondence observes that the \emph{proof}
% of a proposition $A$ can be reinterpreted as a \emph{program} belonging to the
% type $A$.
% This aspect of the correspondence is called the \emph{proofs-as-programs}
% principle.
% Typechecking a program then corresponds to checking that the proof that it
% represents indeed does prove the proposition that one claims it to prove.

% This correspondence explains precisely why a proof assistant can take the form
% of a programming language at all:
% programming and proving are one and the same.
% However, a property one cares greatly about in a logic is its
% \emph{consistency}.
% That is, it ought to be impossible to derive a falsehood.

\Beluga~\cite{Pientka:IJCAR10,Pientka:CADE15} is a proof assistant which
provides sophisticated infrastructure for implementing formal systems based on
the logical framework LF~\cite{Harper93jacm}.
This allows programmers to uniformly specify syntax, inference rules, and
derivation trees using higher-order abstract syntax (HOAS) and relieves users
from having to build custom support for managing variable binding,
renaming, and substitution.

Following the Curry-Howard correspondence, \Beluga{} users develop inductive
metatheoretic proofs about formal systems by writing a total recursive
dependently typed program by pattern matching on derivation trees.
Given that the proof is represented as a program, proof checking amounts to
typechecking the program.
\Beluga{} hence follows in the foot steps of proof checkers such as Automath
\cite{Nederpelt:94}, Agda \cite{Norell:phd07}, and specifically Twelf
\cite{Pfenning99cade}.

While writing a proof as a dependently typed program is a beautiful idea, it
can be quite challenging and cumbersome.
In a dependently typed language, the values of terms can influence the types of
other terms.
Keeping track mentally of this rich type information is nothing short of a tall
order.
This limits the widespread use of dependently typed programming languages for
mechanizing proofs in general.
Hence, many proof assistants in this domain provide some form
of interactivity: for example, Agda \cite{Norell:phd07} supports leaving holes
(question marks) and writing partial programs which can later be refined using a
fixed limited set of interactions.
These holes crucially allow the user to inspect the types of all in-scope
identifiers, which relieves them of the mental tax of dependent types.
However a clear specification and theoretical foundation of how these
interactions transform programs is largely missing.
In \Coq{} \cite{bertot/casteran:2004} users interactively develop a proof using
\emph{tactics}.
Behind the scenes, a sequence of tactic applications is elaborated into a
dependently typed program.
Ideally, applying successfully a tactic to a proof state should only result in a
new valid, consistent proof state, but this isn't always the case: user-defined
tactics for \Coq{} constructed in the Ltac language \cite{Delahaye:LPAR00} are
mostly unconstrained; it is \Coq's typechecker that verifies post hoc that the
program generated by the tactics is valid.
A common additional caveat of tactic languages is that often, the resulting
proof script is brittle and unreadable.

This thesis presents the design and implementation of \Harpoon, an
interactive proof environment built on top of \Beluga, where programmers develop
proofs using a fixed set of tactics.
The user invokes an action on a subgoal in order to eliminate it, possibly
introducing new subgoals in doing so.
Our fixed set of tactics is largely inspired by similar systems such as
Abella \cite{Gacek:IJCAR08} and Coq, supporting introduction of assumptions,
case-analysis, and inductive reasoning, as well as both forward and backward
reasoning styles.
As \Harpoon{} is built on top of \Beluga{}, a tactic can also refer to a
Beluga{} program to provide an explicit proof witness to justify a proof step.
The ability to seamlessly mix programming with command-driven interactive
theorem proving is particularly useful when appealing to a lemma and switching
between proving and programming.
Finally, successful tactic application is guaranteed to transform a valid proof
state into another valid proof state.
%
\Harpoon's command-driven front-end generates automatically as a result a proof
script that reflects the subgoal structure.
We think of a proof script as an intermediate proof representation language to
facilitate translation to other formats, such as into (executable) \Beluga{}
programs as shown in this thesis or perhaps eventually into a human-readable
proof format. My specific contributions are the following:

\begin{figure}[t]
  \centering
  \begin{tikzpicture}

 \node[] (Box0) {
   \begin{tabular}{p{2cm}}
Specification \\[0.5em] \textbf{Statement}
   \end{tabular}
};

 \node[node distance=2.6cm, right= of Box0, block] (Box1) {
   \begin{tabular}{p{4cm}}
     Specification \\[0.5em] \textbf{Statement}
   \end{tabular}
   \nodepart{two} Proof  script};

 \node[node distance=2cm, right= of Box1, block] (Box2) {
   \begin{tabular}{p{4cm}}
     Specification \\[0.5em] \textbf{Statement}
   \end{tabular}
   \nodepart{two} Program};

 \path[pil,draw] (Box0) edge node[above] (label) {action} (Box1);
 \path[pil,draw] (Box1) -- (Box2);

\end{tikzpicture}

  \caption{\Harpoon{} Design Overview}
  \label{fig:harpoon}
\end{figure}

\begin{itemize}
\item
  We present the design and implementation of \Harpoon, an
  interactive command-driven front-end of \Beluga{} for mechanizing
  meta-theoretic proofs.
  Starting from a user-specified theory (including both its syntax and its
  judgments), users interactively develop metatheoretic proofs using tactics.
  In tutorial style, we demonstrate \Harpoon{} to interactively
  develop a proof in Chap.~\ref{chap:example} by way of giving a whirlwind tour of
  the main supported tactics in \Harpoon.

\item
  In Chap.~\ref{chap:theory}, we describe a logical foundation for interactive
  proof development. To that end, we explain first in Sec.~\ref{sec:harpoon} a
  proof script language that reflects the proof structure laid out by the
  user. This language clearly separates forwards and backwards reasoning.
  Then, we explain formally the interactive tactics and their connection to proof
  scripts. We prove soundness of interactive proof construction, and give a
  translation from proof scripts into \Beluga{} programs, showing that this
  translation is type-preserving.
  This justifies that proof scripts indeed represent proofs and form a valid way
  to develop metatheory, and also allows proof scripts to not only be
  typechecked, but also executed as programs.
  Fig.~\ref{fig:harpoon} summarizes the connection among tactics, proof
  scripts, and \Beluga{} programs.

\item
  We characterize and reason about incomplete programs using
  \emph{contextual types}.
  A variable of such a type, which we call a \emph{subgoal variable}, represents a
  hole in the proof, i.e. a statement to prove together with a set of available
  assumptions.
  My formalism of incomplete proofs is such that holes are independent of each
  other and may be solved in any order.
  We show that incremental proof development amounts of successively applying
  contextual substitutions to eliminate a subgoal variable, while possibly
  introducing new ones.

\item
  \Harpoon{} is implemented as part of \Beluga{} and is documented together with
  \Beluga{} at \url{https://beluga-lang.readthedocs.io/}.
  We have used \Harpoon{} for a range of representive examples from the
  \Beluga{} library, in particular type safety proofs for
  MiniML, normalization proofs for the simply-typed lambda calculus
  \cite{Cave:MSCS18},
  % a higher-order solution to POPLMark \cite{Pientka:TPHOLs07},
  benchmarks for reasoning about binders
  \cite{Felty:MSCS17,Felty:orbi-survey},
  and the recent POPLMark Reloaded challenge \cite{POPLMarkReloaded:19}.
  These examples cover a wide range of aspects that arise in the
  proof development such as complex reasoning with and about contexts,
  context schemas, substitutions, and variables.
\end{itemize}

%%% Local Variables:

%%% TeX-master: "../main.tex"
%%% End:
