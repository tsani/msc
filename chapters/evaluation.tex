\chapter{Evaluation}
\label{sec:evaluation}

One should be able to use \Harpoon{} to prove anything that one could prove in
\Beluga. A proper completeness theorem is for now too complex\footnotemark, so
instead we have replicated a number of case studies originally proven as
functional programs in \Beluga. (See table~\ref{table:evaluation}.)
\footnotetext{%
  \Beluga's support for deep pattern matching complicates a potential
  completeness proof.
}

\begin{table}[h]
  \centering
  \begin{tabular}{%
    m{0.4\textwidth} m{0.4\textwidth} %m{0.1\textwidth}
    }
    Case study
    & Main feature tested
    % & Remarks
    \\ \hline
    %%%%%%%%%%%%%%%%%%%%%%%%%%%
    % Examples from \textit{a framework for defining logics}
    % & Unboxing computational variables
    % & \multicolumn{1}{c |}{\---}
    % \\
    %%%%%%%%%%%%%%%%%%%%%%%%%%%
    % Length of context
    % & Splitting on context
    % & \multicolumn{1}{c |}{\ref{itm:evaluation-remark1}}
    % \\
    %%%%%%%%%%%%%%%%%%%%%%%%%%%%
    MiniML value soundness
    & Automatic solving of trivial goals
    % & \multicolumn{1}{c}{\ref{itm:evaluation-remark1}}
    \\
    %%%%%%%%%%%%%%%%%%%%%%%%%%%%
    MiniML compilation completeness
    & Unboxing program variables
    % & \multicolumn{1}{c}{\ref{itm:evaluation-remark3}}
    \\
    %%%%%%%%%%%%%%%%%%%%%%%%%%%%
    STLC type preservation
    & Automatic solving of trivial goals
    % & \multicolumn{1}{c}{\ref{itm:evaluation-remark5}}
    \\
    %%%%%%%%%%%%%%%%%%%%%%%%%%%%
    STLC type uniqueness
    & Open term manipulation
    % & \multicolumn{1}{c}{\ref{itm:evaluation-remark5}}
    \\
    %%%%%%%%%%%%%%%%%%%%%%%%%%%%
    STLC weak normalization
    & Advanced splitting
    % & \multicolumn{1}{c}{\ref{itm:evaluation-remark5}}
    \\
    %%%%%%%%%%%%%%%%%%%%%%%%%%%%
    STLC strong normalization \cite{POPLMarkReloaded:19}
    & Large development
    % & \multicolumn{1}{c}{\ref{itm:evaluation-remark5}}
    \\
    %%%%%%%%%%%%%%%%%%%%%%%%%%%%
    STLC alg. equality completeness \cite{Cave:MSCS18}
    & Large development
    % & \multicolumn{1}{c}{\ref{itm:evaluation-remark5}}
    \\
    %
  \end{tabular}

  \caption{%
    Summary of proofs ported to \Harpoon{} from \Beluga.
  }
  \label{table:evaluation}
\end{table}

The first four examples are purely syntactic arguments that proceed by
straightforward induction.
The remaining examples involve more sophisticated features from \Beluga's
computation language such as inductive and stratified types used to encode
logical relations.

Recreating these case studies in \Harpoon{} was straightforward and provided us
with insight as to future work regarding automation. In the syntactic case
studies, proofs tend to proceed by case analysis on the induction variable,
inverting any other assumptions when possible, invoking available induction
hypotheses, and applying a few inference rules. This general recipe could be
automated in whole or in part to simplify the development of similar simple
proofs.

% \begin{enumerate}
% \item\label{itm:evaluation-remark1}
%   All solve commands include only two or less terms; with more automation tactic this example may be solved automatically.
% \item\label{itm:evaluation-remark3}
%   This uses a lot of unbox tactic. Providing a way to automate unbox tactic may be helpful.
% \item\label{itm:evaluation-remark5}
%   Auto invert tactic would simplify this example.
% \end{enumerate}

%%% Local Variables:
%%% mode: latex
%%% TeX-master: "../main.tex"
%%% End:
