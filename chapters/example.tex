\chapter{Proof Development in \Harpoon}
\label{chap:example}

We introduce the main features of \Harpoon{} by considering two
lemmas that play a central role in proof of weak normalization of
the simply-typed lambda calculus.
First, the Termination Property states that if
well-typed term \lstinline!M'! halts and \lstinline!M!
reduces to \lstinline!M'!, then \lstinline!M'! halts.
Second, the Backwards Closed Property states that if a well-typed term
\lstinline!M'! is reducible and \lstinline!M! reduces to \lstinline!M'!, then
\lstinline!M! is also reducible.

\section{Initial setup: encoding the language}
\label{sec:setup-1}
We begin by defining the simply-typed lambda-calculus in the logical
framework LF \cite{Harper93jacm} using an intrinsically typed
encoding.
In typical HOAS style, lambda abstraction takes an LF function representing the
abstraction of a term over a variable. There is no case for variables, as they
are treated implicitly.
We remind the reader that this is a weak, representational function space~--~%
there is no case analysis or recursion, so only genuine lambda terms can
be represented.
%
\begin{center}
\begin{tabular}{ll}
\begin{lstlisting}
LF tp : type =
  | unit: tp
  | arr : tp -> tp -> tp;
\end{lstlisting}
  &
\begin{lstlisting}
LF tm : tp -> type =
  | lam : (tm T1 -> tm T2) -> tm (arr T1 T2)
  | app : tm (arr T1 T2) -> tm T1 -> tm T2;
\end{lstlisting}
\end{tabular}
\end{center}
%
Free variables such as \lstinline!T1! and \lstinline!T2! are
implicity universally quantified (see \cite{Pientka:JFP13})
and programmers subsequently do not supply arguments for
implicitly quantified parameters when using a constructor.

% In typical higher-order abstract syntax  fashion, the \lstinline!lam!
% constructor takes  an LF function,
% \lstinline!tm A -> tm B!,
% representing  the abstraction of a
% term over a variable.
% There is no case for variables, as they are treated implicitly.
% We
% remind the reader that this is a weak, representational function space
% – there is no case analysis or recursion, and hence only genuine lambda terms can be repre-sente



% The \lstinline!schema! construct declares a new \emph{schema} \lstinline!ctx!:
% whereas types classify values, schemas classify contexts in \Beluga. If we view
% a context as a map whose domain is a set of variables, then a schema declaration
% specifies the codomain of the contexts belonging to that schema.

With the syntax out of the way, we define a small-step operational semantics for
the language. For simplicity, we use a call-by-name reduction
strategy and do not reduce under lambda-abstractions.
%
\begin{center}
  \begin{tabular}{ll}
\begin{lstlisting}
LF step : tm T -> tm T -> type =
  | s_app   : step M M' →
              step (app M N) (app M' N)
  | s_beta : step (app (lam M) N) (M N);
\end{lstlisting}
    &
\begin{lstlisting}
LF steps : tm T → tm T → type =
  | next : step M M' → steps M' N →
            steps M N
  | refl : steps M M;
\end{lstlisting}
  \end{tabular}
\end{center}
%
Notice in particular that we use LF application to encode the
object-level substitution in the \lstinline!s_beta! rule.
We define a predicate \lstinline!val: tm T -> type! on well-typed terms expressing what it
means to be a value: \lstinline!v_lam: val (lam M)!.
Last, we define a notion of termination: a term halts if it reduces to a value. This is captured by the constructor \lstinline!halts/m!.
\begin{center}
  \begin{tabular}{l}
\begin{lstlisting}
LF halts : tm T → type = halts/m : val V → steps M V → halts M;
\end{lstlisting}
  \end{tabular}
\end{center}

\section{%
  Termination Property: tactics \ttintros, \ttsplit, \ttunbox, and \ttsolve%
}
\label{sec:halts-step}

As the first short lemma, we show the Termination Property, that if
\lstinline!M'! is known to halt and \lstinline!steps M M'!, then \lstinline!M!
also halts.
We start our interactive proof session by loading the signature and defining the
name of the theorem and the statement that we want to prove.

\begin{lstlisting}[language=Beluga, deletekeywords={of}]
  Name of theorem (empty name to finish): halts_step
  Statement of theorem: [ |- step M M'] -> [ |- halts M'] -> [ |- halts M]
\end{lstlisting}

\Beluga{} is a proof environment in which an encoded theory is clearly separated
from its metatheory.
LF objects encoding the syntax or judgments from a theory are embedded within
\Beluga{} using the ``box'' syntax \lstinline![ |- ]!.
Furthermore, we embed such LF objects together with the LF context in which they
are meaningful \cite{Pientka:POPL08,Pientka:PPDP08,Nanevski:ICML05}.
We call such an object paired with its context a \emph{contextual object}.
In this example, the LF context, written on the left of \lstinline!|-!, is
empty as we consider closed LF objects.

Whereas a judgment of an encoded
theory is represented as an LF type, a metatheoretic statement is represented as
a \Beluga{} type. As is often the case, implications are modelled using simple
functions written with \lstinline!->!.
As before, the free variables \lstinline!M! and \lstinline!M'! are implicitly
bound by $\Pi$-types at the outside, which correspond to universal
quantification.
In terms of expressiveness, \Beluga{} is comparable to a first-order logic with
fixed points together with LF as an index domain.

With theorem configuration out of the way, the proof begins with a single
subgoal whose type is simply the statement of the theorem under no
assumptions.
Since this subgoal has a function type, \Harpoon{} will automatically apply the
\lstinline!intros! tactic:
first, the (implicitly) universally quantified variables
\lstinline!M!, \lstinline!M'! are added to the meta-context;
second, the assumptions \lstinline!s : [|- step M M']! and
\lstinline!h : [|- halts M']! are added to the program context.
Observe that since \lstinline!M! and \lstinline!M'! have type \lstinline!tm T!,
\lstinline!intros! also adds \lstinline!T! to the meta-context, although it is
implicit in the definitions of \lstinline!step! and \lstinline!halts! and is not
visible.
(See \Harpoon{} example ~\ref{fig:halt-harpoon} Step \ref{step:hs:split-h}.)

\begin{HarpoonProof}[ht] % halts_step proof: three steps
  \NextHarpoonProof
\centering

\begin{tabular}{p{.30\textwidth}|p{.30\textwidth}|p{.36\textwidth}}
  \step{hs:split-h} & \step{hs:unbox} & \step{hs:solve} \tabularnewline
\begin{lstlisting}
Meta-context:
  T : ( |- tp)
  M : ( |- tm T)
  M' : ( |- tm T)




Computational context:
  s : [ |- step M M']
  h : [ |- halts M']
\end{lstlisting}
\linewidthline
\begin{lstlisting}
[ |- halts M]
> split h
\end{lstlisting}
    &
\begin{lstlisting}
Meta-context:
  T : ( |- tp)
  M : ( |- tm T)
  M' : ( |- tm T)
  M2 : ( |- tm T)
  S : ( |- steps M' M2)
  V : ( |- val M2)

Computational context:
  s : [ |- step M M']
  h : [ |- halts M']
\end{lstlisting}
\linewidthline
\begin{lstlisting}
[ |- halts M]
> unbox s as S'
\end{lstlisting}
&
\begin{lstlisting}
Meta-context:
  T : ( |- tp)
  M : ( |- tm T)
  M' : ( |- tm T)
  M2 : ( |- tm T)
  S : ( |- steps M' M2)
  V : ( |- val M2)
  S' : ( |- step M M')
Computational context:
  s : [ |- step M M']
  h : [ |- halts M']
\end{lstlisting}
\linewidthline
\begin{lstlisting}
[ |- halts M]
> solve
  [|-halts/m (next S' S) V]
\end{lstlisting}

\end{tabular}

\caption{%
  Interactive session of the proof for the
  \lstinline!halts_step! lemma.
}
\label{fig:halt-harpoon}

%%% Local Variables:
%%% TeX-master: "../main.tex"
%%% End:

\end{HarpoonProof}

The proof proceeds by inversion on \lstinline!h!.
Using the \lstinline!split! tactic, we add the two new
assumptions \lstinline!S:(|- steps M' M2)! and
\lstinline!V:(|- val M2)!
to the meta-context. %
(See \Harpoon{} example~\ref{fig:halt-harpoon} Step~\ref{step:hs:split-h}.) %
%Alternatively, since we expect there to be only one case as we reason here by
%inversion, we could have used the \lstinline!invert! tactic to capture this
%intent. It is identical to \lstinline!split! internally, but it checks that a
%unique case is produced by splitting.
%
To build a proof for \lstinline![|- halts M]!, we need to show that
there is a step from \lstinline!M! to some value \lstinline!M2!. To
build such a derivation, % using the LF constant \lstinline!next!,
we use first the \lstinline!unbox! tactic on the computation-level
assumption \lstinline!s! to obtain an assumption \lstinline!S'! in the
meta-context which is accessible to the LF layer.
(See \Harpoon{} example~\ref{fig:halt-harpoon} Step~\ref{step:hs:unbox}.)
%
%
Finally, we can finish the proof by supplying the term
\lstinline![ |- halts/m (next S' S) V]!
% , which is a proof for \lstinline![ |- halts M]!.
(See \Harpoon{} example~\ref{fig:halt-harpoon} Step~\ref{step:hs:solve}.)


\section{Setup continued: reducibility}
\label{sec:setup-2}
We now proceed to define normalization using using reducibility candidates
\cite{GirardLafontTaylor:proofsAndTypes}.
In essence, we define a set of terms \emph{reducible} at a type \lstinline!T!. %
All reducible terms are required to halt, and terms reducible at an arrow type
are required to produce reducible output given reducible input.
Concretely,  a term
\lstinline!M! is reducible at type \lstinline!(arr T1 T2)!,
 if for all
terms \lstinline!N:tm T1! where \lstinline!N! is reducible at type
\lstinline!T1!, then \lstinline!(app M N)! is reducible at type \lstinline!T2!.
%This is
%sometimes phrased as ``if good things go in, then good things come out.'' %
Reducibility cannot be directly encoded at the LF layer, as it is not
merely describing the syntax tree of an expression or derivation.
Instead, we encode the set of reducible terms using the stratified
type \lstinline!Reduce! which is recursively defined on the type
\lstinline!T! in \Beluga{} (see \cite{JacobRao:stratified2018}). Note
that we write \lstinline!{ }! for explicit universal quantification.

\begin{lstlisting}
stratified Reduce : {T : [⊢ tp]} [⊢ tm T] → type =
  | Unit: [⊢halts M] → Reduce [⊢unit] [⊢M]
  | Arr : [⊢halts M] →
           ({N:[⊢tm T1]} Reduce [⊢T1] [⊢N] → Reduce [⊢T2] [⊢app M N])
        → Reduce [⊢arr T1 T2] [⊢M];
\end{lstlisting}

\section{%
  Backwards Closed Property: tactics \ttmsplit, \ttsuffices, and \ttby%
}

We now consider one of the key lemmas in the weak normalization proof,
called the backwards closed lemma, i.e.  if \lstinline!M'! is
reducible at some type \lstinline!T! and \lstinline!M! steps to
\lstinline!M'!, then \lstinline!M! is also reducible at \lstinline!T!.

We prove this lemma by induction on \lstinline!T!. This is
specified by referring to the position of the induction variable in
the statement.

\begin{lstlisting}[gobble=2, deletekeywords={of}]
    Name of theorem: bwd_closed
    Statement of theorem: {T : [|-tp]} {M : [|-tm T]} {M' : [|-tm T]}
             [|-step M M'] -> Reduce [|-T] [|-M'] -> Reduce [|-T] [|-M]
    Induction order: 1
\end{lstlisting}

%The sequence of proof states is shown in \Harpoon{} Proof
%~\ref{fig:bwd-closed-harpoon} and \Harpoon{} Proof
%~\ref{fig:bwd-closed-harpoon-step}.
%
After \Harpoon{} automatically
introduces the meta-variables \lstinline!T!, \lstinline!M!,
and \lstinline!M'! together with an assumption
\lstinline!s : [|-step M M']! and \lstinline!r : Reduce [|-T] [|-M']!,
we use \lstinline!msplit T! to split the proof into two cases (see
\Harpoon{} Proof~\ref{fig:bwd-closed-harpoon}
Step~\ref{step:bwd:msplit-T}). Whereas \lstinline!split! case analyzes
a \Beluga{} type, \lstinline!msplit! considers the cases for a
(contextual) LF type. In reality, \lstinline!msplit! is syntactic sugar for a
more verbose use the ordinary \lstinline!split! tactic.

%Had we used the ordinary \lstinline!split! tactic, we would have written
%\lstinline!split [|- T]!; the \lstinline!split! tactic requires a
%\emph{computational} term, so we would use a box to embed the LF term
%\lstinline!A! (together with the empty LF context) in the computation
%language. The tactic \lstinline!msplit! instead infers the necessary LF context,
%at the expense of allowing splitting on metavariables only.

\begin{HarpoonProof}[htb]
  \NextHarpoonProof
\centering

\newcommand{\cwidthOne}{0.33\textwidth}
\newcommand{\cwidthTwo}{0.33\textwidth}
\newcommand{\cwidthThree}{0.34\textwidth}

\newenvironment{eventable}{%
  \begin{tabular}{p{\cwidthOne}|p{\cwidthTwo}|p{\cwidthThree}}%
    }{%
  \end{tabular}
}

\begin{eventable}
  % STEPS 1 2 3
  \step{bwd:msplit-T} & \step{bwd:b-invert-r} & \step{bwd:b-solve} \tabularnewline

% STEP 1
\begin{lstlisting}
Meta-context:
  T : ( |- tp )
  M : ( |- tm T )
  M' : ( |- tm T )
Computational context:
  s : [|-step M M']
  r : Reduce [|-T] [|-M']
$~$
\end{lstlisting}

\linewidthline

\begin{lstlisting}
Reduce [|-T] [|-M]
> msplit T
\end{lstlisting}
  &

% STEP 2
\begin{lstlisting}
Meta-context:
  M : ( |- tm b )
  M': ( |- tm b )

Computational context:
  s : [|-step M M']
  r : Reduce [|-b] [|-M']
$~$
\end{lstlisting}

\linewidthline

\begin{lstlisting}
Reduce [|-b] [|-M]
> split r
\end{lstlisting}
&
% STEP 3
\begin{lstlisting}
Meta-context:
  M : ( |- tm b )
  M': ( |- tm b )

Computational context:
  s : [|-step M M']
  h': [|-halts M' ]
  r : Reduce [|-b] [|-M']
\end{lstlisting}
\linewidthline
\begin{lstlisting}
Reduce [|-b] [|-M]
> by halts_step s h'
  as h;
  solve Unit h
\end{lstlisting}
\end{eventable}



\caption{%
%   Interactive session
% Part A of the
  Backwards closed lemma. Step 1: Case analysis of the type
  \lstinline!T!; Steps 2 and 3: Base case (\lstinline!T!~$=$~\lstinline!b!).
%  (This session continues in example~\ref{fig:bwd-closed-harpoon-step}.)
}
\label{fig:bwd-closed-harpoon}

%%% Local Variables:
%%% TeX-master: "../main.tex"
%%% End:

\label{fig:bwclosed-base}
\end{HarpoonProof}

The case for \lstinline!T!~$=$~\lstinline!b! is
straightforward (see \Harpoon{} Proof~\ref{fig:bwclosed-base} Step
\ref{step:bwd:b-invert-r} and \ref{step:bwd:b-solve}). First, we use
the \lstinline!split! tactic to invert the premise
\lstinline!r : Reduce [|- b] [|- M']!.
%
Then, we use the \lstinline!by! tactic to invoke the
\lstinline!halts_step! lemma (see Sec.~\ref{sec:halts-step}) to obtain
an assumption \lstinline!h : [|- halts M]!. %
We \lstinline!solve! this case by supplying the term \lstinline!Unit h! %
(\Harpoon{} Proof~\ref{fig:bwd-closed-harpoon} Step~\ref{step:bwd:b-solve}). %

\begin{HarpoonProof}[htb]
  % \NextHarpoonProofCont
\tightlistings
\centering

\begin{tabular}{p{5.5cm}|@{~~}p{7cm}}
  \step{bwd:arr-invert} & \step{bwd:arr-by-suffices}
  \tabularnewline

  \begin{lstlisting}
Meta-context:
  T1 : (|-tp)
  T2 : (|-tp)
  M  : (|-tm (arr T1 T2))
  M' : (|-tm (arr T1 T2))
Computational context:
  s  : [|-step M M']
  r  : Reduce [|-arr T1 T2][|-M']
$~$
$~$
$~$
\end{lstlisting}
\linewidthline
\begin{lstlisting}
Reduce [|-arr T1 T2][|-M]
> split r
  \end{lstlisting}
  &
  \begin{lstlisting}
Meta-context:
  T1 : (|-tp)
  T2 : (|-tp)
  M  : (|-tm (arr T1 T2))
  M' : (|-tm (arr T1 T2))
Computational context:
  s  : [|-step M M']
  rn : {N : ( |-tm T)} Reduce [|-N][|-T]
    -> Reduce [|-T2][|-app M' N]
  h' : [|-halts M']
  r  : Reduce [|-arr T1 T2][|-M']
\end{lstlisting}
\linewidthline
\begin{lstlisting}
Reduce [|-arr T1 T2][|-M]
> suffices by Arr
1> [|-halts M]
2> {N : ( |-tm T1)}Reduce [|-T1][|-N]
   -> Reduce [|-T2][|-app M N]
  \end{lstlisting}
\end{tabular}


\caption{Backwards closed lemma: Step Case}
\label{fig:bwd-closed-harpoon-step-part}

%%% Local Variables:
%%% TeX-master: "../main.tex"
%%% End:

\end{HarpoonProof}

In the case for \lstinline!T!~$=$~\lstinline!arr T1 T2!, we begin
similarly by inversion on \lstinline!r! using the \lstinline!split!
tactic (\Harpoon{} Proof~\ref{fig:bwd-closed-harpoon-step-part} Step~\ref{step:bwd:arr-invert}).
% to obtain the assumptions
%\begin{lstlisting}
%  h' : [|-halts M']
%  rn : {N : [|-tm T]} Reduce [|-T] [|-N] -> Reduce [|-T1] [|-app M' N]
%\end{lstlisting}
%
%
%
We observe that the goal type is %
\lstinline!Reduce [|- arr T1 T2] [|- M]!, %
which can be produced by using the \lstinline!Arr! constructor if we can
construct a proof for each of the user-specified types,
\lstinline![⊢halts M]! and
\lstinline!{N:[⊢tm T1]} Reduce [⊢T1] [⊢N] → Reduce [⊢T2] [⊢app M N]!.
%
Such \emph{backwards reasoning} is accomplished via the \lstinline!suffices!
tactic. The user supplies a term representing an implication whose conclusion is
compatible with the current goal and proceeds to
prove its premises as specified
(see \Harpoon{} Proof \ref{fig:bwd-closed-harpoon-step-part} Step \ref{step:bwd:arr-by-suffices}).

To prove premise \lstinline!1>!, we
apply the
\lstinline!halts_step! lemma
% with \lstinline!h'! and \lstinline!s!
%  to
% deduce \lstinline!h : [|- halts M]!
(\Harpoon{} Proof~\ref{fig:bwd-closed-harpoon-step} Step~\ref{step:bwd:halt}). %
%
As for premise \lstinline!2>!, \Harpoon{} first automatically introduces the
variable \lstinline!N:(|-tm T1)! and the assumption
\lstinline!r1:Reduce [|-T1] [|-N]!, so it remains to show
\lstinline!Reduce [|-T2] [|-app M N]!.
We deduce \lstinline!r':Reduce [|-T2] [|-app M' N]!
using the assumption
\lstinline!rn!. Using \lstinline!s:[|- step M M']!, we build a
derivation \lstinline!s':[|-step (app M N) (app M' N)]! using
\lstinline!s_app!.
Finally, we appeal to the induction hypothesis. Using the \lstinline!by!
tactic, we write out and refer to the recursive call to complete the proof
(\Harpoon{} Proof~\ref{fig:bwd-closed-harpoon-step} Step~\ref{step:bwd:final}).
% \Harpoon{} uses this variable to automatically solve the goal.

Note that \Harpoon{} allows users to use underscores to stand for arguments that
are uniquely determined
(see \Harpoon{} Proof~\ref{fig:bwd-closed-harpoon-step}
Step~\ref{step:bwd:final}).
We enforce that these underscores stand for uniquely determined objects in order
to guarantee that the contexts and the goal type of every subgoal be
closed. This ensures modularity: solving one subgoal does not affect any
other open subgoals.
% For example, when using the \lstinline!suffices!
% tactic where the user supplies a term that would solve a given goal
% provided we have proofs for the listed subgoals, each of those yet to
% be proven subgoals may not contain any unknowns (holes). For example,
% in the base case of the backwards closed lemma (see \Harpoon{}
% Proof~\ref{fig:bwclosed-base} Step \ref{step:bwd:b-solve}), one might
% wonder if one could have attempted a backwards reasoning step by the
% action \lstinline!suffices by halts_step!. This would however fail, as
% it is undetermined where \lstinline!M! steps to and the two generated
% subgoals would hence contain a hole. On the other hand the action
% \lstinline!suffices by halts_step s! would succeed and generate the
% subgoal \lstinline!halts M'!. Enforcing that no generated subgoal can
% contain unknowns (holes) is a sensible restriction, as we expect every
% proof step to reach a concrete fully determined proof state, i.e. one
% that does not contain unknowns.


\begin{HarpoonProof}[h]
  % \NextHarpoonProofCont
\tightlistings
\centering

\newcommand{\cwidthOne}{0.33\textwidth}
\newcommand{\cwidthTwo}{0.33\textwidth}
\newcommand{\cwidthThree}{0.34\textwidth}

\newenvironment{eventable}{%
  \begin{tabular}{p{\cwidthOne}|p{\cwidthTwo}|p{\cwidthThree}}%
    }{%
  \end{tabular}
}


% \begin{tabular}{p{6.5cm}|@{~~}p{8cm}}
%   \step{bwd:arr-invert} & \step{bwd:arr-by-lemma}
%   \tabularnewline

%   \begin{lstlisting}

\begin{tabular}{p{6.4cm}|@{~}p{8cm}}
\step{bwd:halt} & \step{bwd:final}
 \tabularnewline

  \begin{lstlisting}
Meta-context:
  T1 : (|-tp)
  T2 : (|-tp)
  M  : (|-tm (arr T1 T2))
  M' : (|-tm (arr T1 T2))

Computational context:
  s  : [|-step M M']
  rn : {N : ( |-tm T)} Reduce [|-N][|-T]
   -> Reduce [|-T2] [|-app M' N]
  h' : [|-halts M']
  r  : Reduce [|-arr T1 T2][|-M']
$~$
\end{lstlisting}
\linewidthline
\begin{lstlisting}
[|-halts M]
> by halts_step s h'  as h
  \end{lstlisting}

&

\begin{lstlisting}
Meta-context:
  T1 : (|-tp)
  T2 : (|-tp)
  M  : (|-tm (arr T1 T2))
  M' : (|-tm (arr T1 T2))
  N  : (|-tm T1)
Computational context:
  s  : [|-step M M']
  rn : {N : ( |-tm T)} Reduce [|-N][|-T]
    -> Reduce [|-T2] [|-app M' N]
  h' : [|-halts M']
  r  : Reduce [|-arr T1 T2][|-M']
  r1 : Reduce [|-T1] [|- N]
\end{lstlisting}
\linewidthline
\begin{lstlisting}
Reduce [|-T2] [|-app M N]
>  by (rn [|-N] r1) as r';
   unbox s as S; by [|-s_app S] as s';
   by (bwd_closed [|-T2] _ _ s' r')
   as ih
\end{lstlisting}
\end{tabular}


\caption{Backwards closed lemma: Step Case -- continued}

\label{fig:bwd-closed-harpoon-step}

%%% Local Variables:
%%% TeX-master: "../main.tex"
%%% End:

\end{HarpoonProof}

Using the explained tactics, one can now prove the fundamental lemma and the weak
normalization theorem. For a more comprehensive description of this
proof in \Beluga{} see \cite{Cave:LFMTP13,Cave:MSCS18}.

\section{Additional features}

Besides the tactics already discussed in this section, our
implementation supports several others.
%
Two are variants on \lstinline!split!. First, the \lstinline!invert! tactic
splits on the type of a given term, but checks that the split produces a unique case. Second, the
\lstinline!impossible! tactic verifies that the split produces no cases, so the
supplied term's type is empty.

The \lstinline!strengthen! tactic can be used to strengthen the contextual type
of a given declaration according to a type subordination analysis \cite{something}.
This tactic is essential in the completeness proof for algorithmic equality
\cite{Cave:MSCS18}.

We also support a number tactics that do not contribute to the elaboration of
the proof, called \emph{administrative tactics.}
Many of these are for navigating and listing theorems and subgoals.
Besides navigation commands, we include an \lstinline!undo!
tactic for rolling back previous steps in a proof.

Our implementation also performs some rudimentary automation to detect
available assumptions that match the current goal type. Already, this is quite
convenient as it automatically eliminates certain trivial subgoals from proofs.

% Our
%implementation also automatically invokes \ttintros{} when the goal is
%a function type.

%\subsection{The Main Theorem: Fundamental Lemma}
% Proving that all well-typed terms halt is now done in two parts.  Clearly, any
% reducible term halts by definition. Therefore, the challenging part of the
% argument is to show that all well-typed terms \lstinline!M : tm A! are
% reducible, i.e. \lstinline!Reduce [ |- A] [ |- M]!. As well-typed terms may be
% open, we need to generalize the statement: ``for any reducible substitution
% $\sigma$, all well-typed terms $\Gamma \vdash$ \lstinline!M : tm A! are
% reducible, i.e.  \lstinline!Reduce [|- A] [|- M[sigma]]!'', where a
% \emph{reducible substitution} is a map from variables to reducible terms.  This
% is elegantly handled in \Beluga{} using the built-in support for substitutions,
% but does not require any new tactics.

% The fundamental lemma can be proven interactively with the previously
% described tactics and lemmas. For a more comprehensive description of
% this proof see \cite{Cave:LFMTP13,Cave:MSCS18}.
% In addition, we have
% also considered the strong normalization proof for STLC. While the set
% up is somewhat more complex, the set of previously described tactics suffices.

%%% Local Variables:
%%% TeX-master: "../main.tex"
%%% End:
