\usepackage[dvipsnames]{xcolor}
\usepackage[lighttt]{lmodern}

\usepackage{array}
\usepackage{graphicx}
\usepackage[utf8]{inputenc}
\usepackage{hyperref}
\renewcommand\UrlFont{\color{blue}\rmfamily}
\usepackage{proof}
\usepackage{amsthm}
\usepackage{amsmath}
\usepackage{amssymb}
\usepackage{stmaryrd}
\usepackage{float}
\usepackage{hhline}
\usepackage{underscore}

%% To make clicking on links to figures, tables, etc. take us to the top of
%% the *figure* instead of to the top of the *caption* !
\usepackage[all]{hypcap}

\usepackage{setspace}
\onehalfspacing
% \renewcommand{\baselinestretch}{1.5}

%%% TiKZ begin

\usepackage{tikz} % For trees.
\usetikzlibrary{positioning,shapes.multipart}
\usetikzlibrary{arrows,positioning}
\tikzset{
    %Define standard arrow tip
    >=stealth',
    %Define style for boxes
  block/.style ={draw=black, thin, text width=3cm,align=center, rounded corners, minimum height=2em, draw, shape=rectangle split, rectangle split parts=2,
 inner sep=1ex},
    % Define arrow style
    pil/.style={
           ->,
           thin,
           shorten <=3pt,
           shorten >=3pt,}
       }

%%% TiKZ end

\makeatletter

%%%%%%%%%%%%%%%%%%%%%%%%%%%%%%%%%%%%%%%%%%%%%%%%%%%%%%%%
% LISTINGS
%%%%%%%%%%%%%%%%%%%%%%%%%%%%%%%%%%%%%%%%%%%%%%%%%%%%%%%%
\definecolor{burntumber}{rgb}{0.54, 0.2, 0.14}

\usepackage{listings}
\usepackage{lstcoq}

\newcommand{\nicettfamily}{%
  \fontseries{l}\selectfont
  \footnotesize\ttfamily%
}%

\lstdefinelanguage{Beluga}{%
  morekeywords = [1]{% real keywords
    LF, type, ctype, schema, rec, proof, as, case, by, of, unbox, inductive,
    stratified, %
    % Harpoon commands:
    solve, msplit, suffices, split, by, as, ih, lemma, invert,
    impossible, undo
  },%
  keywordstyle = [1]{%
%    \nicettfamily\bfseries\color{burntumber}%
     \fontseries{b}\selectfont
  },%
  morekeywords = [2]{% constructors are red ...
    unit, arr, c, lam, app, v_lam, v_c, s_app_1, s_app, s_beta, next, refl, halts/m, %
    Unit, Arr %
  },%
  keywordstyle = [2]{\nicettfamily\color{purple!90}},%
  morekeywords = [3]{% types are blue ...
    tp, tm, value, step, steps, halts, Reduce, RedSub, ctx %
  },%
  keywordstyle = [3]{%
    \nicettfamily\color{BlueViolet}
  },%
  alsoletter=/,%
  columns=flexible,%
  sensitive = true,%
  basicstyle=\nicettfamily,%
  mathescape=true,%
  literate={%
    {→}{$\rightarrow$ }1%
    {->}{$\rightarrow$ }1%
    {\\}{$\lambda$}1%
    {⊢}{$\proves\;$}1%
    {|-}{$\proves\;$}1%
    {sigma}{$\sigma$}1%
  }%
}%

\lstset{
  language={Beluga},
  basicstyle=\renewcommand{\baselinestretch}{1}\nicettfamily
}

\newcommand{\tightlistings}{%
  \lstset{%
    belowskip=-0.5\baselineskip,
    aboveskip=0pt
  }
}

%%%%%%%%%%%%%%%%%%%%%%%%%%%%%%%%%%%%%%%%%%%%%%%%%%%%%%%%
% TODO ITEMS
%%%%%%%%%%%%%%%%%%%%%%%%%%%%%%%%%%%%%%%%%%%%%%%%%%%%%%%%

\usepackage{todonotes}
% \setuptodonotes{color=green!40}
\newcommand{\inlinetodo}[1]{\todo[inline]{#1}}

%%%%%%%%%%%%%%%%%%%%%%%%%%%%%%%%%%%%%%%%%%%%%%%%%%%%%%%%
% SYSTEM NAMES
%%%%%%%%%%%%%%%%%%%%%%%%%%%%%%%%%%%%%%%%%%%%%%%%%%%%%%%%

% \mksystemname{foo} creates a command \foo that outputs the text "foo" in
% smallcaps.
\newcommand{\mksystemname}[1]{%
  \expandafter\newcommand\csname #1\endcsname{%
    \textsc{#1}%
  }%
}
\mksystemname{Harpoon}

% I only want Harpoon to be formatted specially.
% All other systems will use ordinary formatting.
\renewcommand{\mksystemname}[1]{%
  \expandafter\newcommand\csname #1\endcsname{%
    #1%
  }%
}%
\mksystemname{Abella}
\mksystemname{Beluga}
\mksystemname{Coq}
\mksystemname{Agda}
\mksystemname{Isabelle}
\mksystemname{Emacs}
\mksystemname{Twelf}
\mksystemname{LCF}
% the slash means we can't use mksystemname
\newcommand{\IsabelleHOL}{\textsc{Isabelle/HOL}}
\mksystemname{Mizar}

% VeriML's object language, lambda-HOL_inductive
\newcommand{\lhol}{$\lambda \mathsf{HOL}_{ind}$}

%%%%%%%%%%%%%%%%%%%%%%%%%%%%%%%%%%%%%%%%%%%%%%%%%%%%%%%%
% MATH / PROOF COMMANDS
%%%%%%%%%%%%%%%%%%%%%%%%%%%%%%%%%%%%%%%%%%%%%%%%%%%%%%%%

% To make a theorem consisting of several mutual statements more compact, this
% command is used before the \begin{enumerate}
\newcommand{\mutualtheorems}{~\\[-2.0em]}

% For long squiggle arrow
\newlength{\LETTERheight}
\AtBeginDocument{\settoheight{\LETTERheight}{I}}
\newcommand*{\longleadsto}[1]{\ \raisebox{0.24\LETTERheight}{\tikz \draw [->,
line join=round,
>=stealth',
decorate, decoration={
  % zigzag, % uncomment for squiggle
  segment length=4,
  amplitude=.9,
  post=lineto,
  post length=2pt
}] (0,0) -- (#1,0);}\ }

% "interactive transition"
\newcommand{\basicitrans}{\ensuremath\longleadsto{0.55}}
\DeclareMathOperator{\itrans}{\mathbin{\basicitrans}}
\newcommand{\itranss}[1]{\mathbin{\overset{#1}{\basicitrans}\hspace{-1ex}^*}\,}
\newcommand{\proves}{\vdash}
\newcommand{\sem}{\vDash}

\newcommand{\chk}{\Longleftarrow}
\newcommand{\syn}{\Longrightarrow}
\newcommand{\oute}{\rightsquigarrow}
% \renewcommand{\implies}{\supset}
\newcommand{\true}{\,\mathsf{true}}
\newcommand{\deriv}{::}
\newcommand{\pprime}{{\prime\prime}}
% substitution for derivations
\newcommand{\dsubst}[1]{[#1]}
\newenvironment{proofcases}{%
  \begin{description}%
    \newcommand{\case}{\item[Case]}%
  }{%
  \end{description}%
}
\newenvironment{prooftable}{%
  \newcommand{\colfrac}{.48}%
  \newcommand{\sz}{\colfrac\textwidth}%
  \begin{tabular}{p{\sz}p{\sz}}%
    }{%
  \end{tabular}%
}

\newcommand{\interp}[1]{\llbracket #1 \rrbracket}
\newcommand{\Union}{\bigcup}
\newcommand{\ass}{{:}}
\newcommand{\der}{\mathcal} % refer to a derivation
% shorthand for something deduced by an abstract derivation
\newcommand{\ded}[2]{%
  \deduce[#2]{#1}{}
}

% To make \vec extend over its argument
\renewcommand{\vec}{\overrightarrow}

% \mkscommand{foo} defines two commands.
% 1. \sfoo which prints foo in serif font (even in math mode!)
% 2. \sfoos which prints foo in serif font followed by a space.
\newcommand{\mkscommand}[1]{%
  \expandafter\newcommand\csname s#1\endcsname{%
    \text{\sf #1}%
  }%
  \expandafter\newcommand\csname s#1s\endcsname{%
    \text{\sf #1}\;%
  }%
}

% Unroll function for decomposition function types.
\mkscommand{unroll}
\mkscommand{roll}
% Coverage function using in splitting
\mkscommand{cov}

\newtheorem{thm}{Theorem}
% \newtheorem*{prop}{Property}
\def\property{\par\medskip\noindent{\bf Properties:} \ignorespaces}
\def\endproperty{}

% We frequently need to refer to a subgoal's type, which is this contextual
% type.
% This command is parametric so you can do e.g. \goal{^\prime} and \goal{_i}
\newcommand{\goalctx}[1]{\Delta#1; \Gamma#1} % just the goal context
\newcommand{\goalwith}[2]{(#1 \vdash #2)} % basic goal type formatting
% \newcommand{\goalclosed}{\square\tau} % closed context
\newcommand{\goalclosed}{(\cdot ; \cdot \vdash \tau)}
\newcommand{\goal}[1]{\goalwith{\goalctx{#1}}{\tau#1}}

% Basic substitution with square brackets.
\newcommand{\subst}[2]{[#1]#2}

% \psubst{P'/g}{P} formats a subgoal substitution of P' for g in P.
\newcommand{\psubst}{\subst}  %[2]{[#1]#2}

% \msubst{theta}{i} formats a modal substitution theta applied to i.
% \newcommand{\msubst}[2]{\llbracket#1\rrbracket #2}
\newcommand{\msubst}[2]{[#1]#2}


% Letter we use to represent an interactive command / action.
\newcommand{\action}{\alpha}
% Sequence of actionseqs
\newcommand{\actionseq}{\bar\action}



% Translation arrow
\newcommand{\transto}{-\!\!\!\!\rightharpoonup}
% To make a longer arrow, I put a minus before the harpoon and nudge the harpoon
% to the left so they overlap. Thanks to Marcel for this trick.

\newcommand{\unif}{\overset{?}{=}}

% grammar alternative
\newcommand{\galt}{\mid}

% Format a BNF grammar.
% Example:
% \begin{grammar}
%   \category{Terms}{t}{%
%     t_1\; t_2 \galt
%     \lambda x.\, t \galt
%     ...
%   }
% \end{grammar}
%
% Each grammar environment can have multiple \category declarations in it.
% The grammar environment does not intelligently split lines for you, so if you
% need to continue a declaration on the following line, use the command
% \continue{%
%   case t of \vec{B} \galt
%   (t_1, t_2) \galt
%   ...
% }
%
\newenvironment{grammar}{%
  \newcommand{\category}[3]{%
    \mbox{##1} & ##2 & ::= & ##3 \\
  }
  \newcommand{\continue}[1]{%
    & & \mid & ##1 \\
  }
  \begin{array}{l@{~}c@{~}r@{~}l}
    }{%
  \end{array}
}

%%%%%%%%%%%%%%%%%%%%%%%%%%%%%%%%%%%%%%%%%%%%%%%%%%%%%%%%
% INFERENCE RULE NAMES
%%%%%%%%%%%%%%%%%%%%%%%%%%%%%%%%%%%%%%%%%%%%%%%%%%%%%%%%

% \mkrulename{\cmd}{name} defines the command \cmd to print the text `name`
% formatted as a name of an inference rule.
\newcommand{\mkrulename}[2]{%
  \expandafter\newcommand{#1}{%
    \text{\sc #2}%
  }%
}

\mkrulename{\irefl}{I-Refl}
\mkrulename{\isingle}{I-Single}

%%%%%%%%%%%%%%%%%%%%%%%%%%%%%%%%%%%%%%%%%%%%%%%%%%%%%%%%
% LANGUAGE SYNTAX
%%%%%%%%%%%%%%%%%%%%%%%%%%%%%%%%%%%%%%%%%%%%%%%%%%%%%%%%

% The 'tt' family of commands are for the INTERACTIVE commands.
% The 'kw' family of commands are for PROOF SCRIPT keywords.
% The 't' family of commands are for BELUGA terms.

% The following commands define helper macros for generating macros to print
% keywords in an appropriate font.
% They all have the same syntax:
% \mkfoocommand{bar} generates two commands:
% \foobar to print ``bar''
% \foobars to print ``bar'' followed by a space (useful in math mode)
% The generated commands will work in math mode and in text mode.

\newcommand{\helpmkcommand}[3]{%
  \text{#3%
    \if#1\relax%
    #2%
    \else%
    #1%
    \fi%
  }%
}

% \mkmkcommand{foo}{\bar}
% generates a command \mkfoocommand such that
% \mkfoocommand{hello} generates a pair of commands
% \foohello and \foohellos
% such that each emits "\bar hello", and the latter a trailing $\;$ to
% facilitate use in math mode.
\newcommand{\mkmkcommand}[2]{%
  \expandafter\newcommand\csname mk#1command\endcsname[2][\relax]{%
    \expandafter\newcommand\csname #1##2\endcsname{%
      \helpmkcommand{##1}{##2}{#2}%
    }%
    \expandafter\newcommand\csname #1##2s\endcsname{%
      \helpmkcommand{##1}{##2}{#2}\;%
    }%
  }
}

\mkmkcommand{tt}{\tt}
\mkmkcommand{kw}{\bf}
\mkmkcommand{t}{\tt}

% Interactive commands
\mkttcommand{solve}
\mkttcommand{intros}
\mkttcommand{split}
\mkttcommand{by}
\mkttcommand{unbox}
\mkttcommand{as}
\mkttcommand{case}
\mkttcommand{msplit}
\mkttcommand{impossible} % implementation-only
\mkttcommand{invert} % implementation-only
\mkttcommand{suffices}
%
%\mkttcommand[suffices$\;$by]{sufficesby}
%\mkttcommand[to$\;$show]{toshow}


% Proof script keywords
\mkkwcommand{solve}
\mkkwcommand{unbox}
\mkkwcommand{intros}
\mkkwcommand{split}
\mkkwcommand{case}
\mkkwcommand{by}
\mkkwcommand{as}
\mkkwcommand{suffices}
\mkkwcommand[suffices$\;$by]{sufficesby}
\mkkwcommand[to$\;$show]{toshow}

% Beluga program keywords
\mktcommand{in}
\mktcommand{let}
\mktcommand[let$\,$box]{letbox}
\mktcommand{case}
\mktcommand{of}
\mktcommand{fn}
\mktcommand{mlam}

\newcommand{\const}[1]{\textbf{#1}}
% Helper for formatting Harpoon hypotheticals.
\newcommand{\hypoth}[2]{\{ #1 \proves #2 \}}

% For Beluga case expression branches, mlam, fn, etc.
\newcommand{\bto}{\Rightarrow}

%%%%%%%%%%%%%%%%%%%%%%%%%%%%%%%%%%%%%%%%%%%%%%%%%%%%%%%%
% JUDGMENTS
%%%%%%%%%%%%%%%%%%%%%%%%%%%%%%%%%%%%%%%%%%%%%%%%%%%%%%%%

% \declarejudgment{judgment}{explanation}
% formats a small table with the judgment on the left and the explanation on the
% right.
% If <judgment> is math, you need to use $...$ yourself.
\newcommand{\declarejudgment}[2]{%
  \begin{tabular}{>{\centering\arraybackslash}m{0.35\textwidth}m{0.649\textwidth}}
    #1 & #2
  \end{tabular}%
}

% Multiline form for big judgments
\newcommand{\declarejudgmentmulti}[2]{%
  \begin{center}
    #1\\[1em]

    #2
  \end{center}
}

%%%%%%%%%% BELUGA type checking and synthesis judgments.

\newcommand{\belchkjudg}[3]{%
  #1 \proves #2 \chk #3
}
\newcommand{\belsynjudg}[3]{%
  #1 \proves #2 \syn #3
}

% Special turnstiles to distinguish proofs from directives in typing rules.
\newcommand{\prfproves}{\proves_\mathbf{P}}
\newcommand{\dirproves}{\proves_\mathbf{D}}

%%%%%%%%%% HARPOON proof script / directive typing judgments

% The contexts come together in the first argument.
\newcommand{\prfchkjudg}[3]{%
  #1 \prfproves #2 \chk #3
}

% As before, contexts come together in the first argument.
\newcommand{\dirchkjudg}[3]{%
  #1 \dirproves #2 \chk #3
}

%%%%%%%%%% HARPOON translation judgments

% Harpoon proof translation judgment.
\newcommand{\prftransjudg}[4]{%
  #1 \prfproves #2 \transto #3 \chk #4
}

% Harpoon directive translation judgment.
\newcommand{\dirtransjudg}[4]{%
  #1 \dirproves #2 \transto #3 \chk #4
}

%%%%%%%%%% INTERACTIVE HARPOON transition judgments
\newcommand{\outa}[2]{(\Lambda #1 .~#2)}
% \itransjudg{goal}{proof}{proof'}{new subgoals}
\newcommand{\itransjudg}[4]{%
  #1 \overset{#2}{\itrans} #3 \proves #4
% #1 \vdash #2 : \outa{#3}{#4}
}

\newcommand{\itrnsjudg}[5]{%
  #1 \proves #3 : #2 \itrans #4 \vdash #5
}

% Interactive transition chain judgment
% \itranssjudg{initial subgoals}{initial proof}{commands}{final
% subgoals}{final proof}

\newcommand{\itranssjudg}[5]{%
  % Just reuse the single-step judgment since we don't really need to use the
  % star arrow
  \itransjudg{#1 \proves #2}{#3}{#4}{#5}
  %#1 \proves #2 \itranss{#3} #4 \proves #5
}

% arg 1 should usually be \goalwith{...}{...}
% this command generates the parens.
\newcommand{\unrolljudg}[4]{%
  g : #1 \mid #2 \proves #4 \oute #3
}

%%%%%%%%%%%%%%%%%%%%%%%%%%%%%%%%%%%%%%%%%%%%%%%%%%%%%%%%
% TIKZ
%%%%%%%%%%%%%%%%%%%%%%%%%%%%%%%%%%%%%%%%%%%%%%%%%%%%%%%%

\usepackage{tikz}
\usetikzlibrary{decorations.pathmorphing, arrows}

% Draws a line across the current box.
% Used to format "inference rules" for Harpoon proof steps.
\newcommand{\linewidthline}{%
  \tikz{\draw (0, 0) -- (\linewidth, 0)}
}

%%%%%%%%%%%%%%%%%%%%%%%%%%%%%%%%%%%%%%%%%%%%%%%%%%%%%%%%
% Example step cross-references
%%%%%%%%%%%%%%%%%%%%%%%%%%%%%%%%%%%%%%%%%%%%%%%%%%%%%%%%

\usepackage{float}

\newcounter{proofnum}
\newfloat{HarpoonProof}{H}{harpoon-proofs}
\floatname{HarpoonProof}{\bfseries \Harpoon{} example}


\setcounter{proofnum}{0}
\newcounter{stepnum}
\setcounter{stepnum}{0}

\newcommand{\NextHarpoonProof}{%
  \setcounter{stepnum}{0}
  \stepcounter{proofnum}%
  % make listings tighter
  \lstset{%
    belowskip=-0.3\baselineskip,
    aboveskip=-0pt
  }
}

\newcommand{\NextHarpoonProofCont}{%
  \stepcounter{proofnum}%
}

% \let\theOldStepnum\thestepnum
% \renewcommand{\thestepnum}{%
%   \number\value{proofnum}.\theOldStepnum%
% }
\newcommand{\labelstep}[1]{%
  % \phantomsection%
  \stepcounter{stepnum}
  \def\@currentlabel{\thestepnum}
  \phantomsection
  \label{step:#1}%
}
\newcommand{\step}[1]{%
  \bfseries
  \labelstep{#1}\centering Step \arabic{stepnum}
}

%%%%%%%%%% MISCELLANEOUS

\newcommand{\meh}[1]{{\color{red}#1}}

%%%%%%%%%% END

\makeatother

%%% Local Variables:
%%% TeX-master: "main.tex"
%%% End:
